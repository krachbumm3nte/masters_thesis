\documentclass[11pt,a4paper,titlepage]{report}
\usepackage[english]{babel}
\usepackage[utf8]{inputenc}
\usepackage{xcolor}
\usepackage{graphicx}
\usepackage{subcaption}
\graphicspath{ {./images/} }
\usepackage[font=small,labelfont=bf]{caption}
\usepackage{listings}
\usepackage{amsmath}
\usepackage{amssymb}
\usepackage[round]{natbib}
\bibliographystyle{apalike}
\usepackage[onehalfspacing]{setspace}
\usepackage{etoolbox}
\AtBeginEnvironment{quote}{\par\singlespacing\small}
\usepackage[top=100pt,bottom=100pt,left=75pt,right=75pt]{geometry}
\usepackage{enumitem}
\usepackage{hyperref}
\usepackage[noabbrev]{cleveref}

\DeclareSymbolFont{letters}{OML}{ztmcm}{m}{it}
\DeclareSymbolFontAlphabet{\mathnormal}{letters}
\pagestyle{headings}


\hypersetup{
    colorlinks=true,
    linkcolor=black,
    citecolor=black,
    urlcolor=blue,
}


\newcommand*\ttvar[1]{\texttt{\expandafter\dottvar\detokenize{#1}\relax}}
\newcommand*\dottvar[1]{\ifx\relax#1\else
  \expandafter\ifx\string_#1\string_\allowbreak\else#1\fi
  \expandafter\dottvar\fi}

\definecolor{codegreen}{rgb}{0,0.6,0}
\definecolor{codegray}{rgb}{0.5,0.5,0.5}
\definecolor{codepurple}{rgb}{0.58,0,0.82}
\definecolor{backcolour}{rgb}{0.95,0.95,0.92}

\lstdefinestyle{mystyle}{
    backgroundcolor=\color{backcolour},   
    commentstyle=\color{codegreen},
    keywordstyle=\color{magenta},
    numberstyle=\tiny\color{codegray},
    stringstyle=\color{codepurple},
    basicstyle=\ttfamily\footnotesize,
    breakatwhitespace=false,         
    breaklines=true,                 
    captionpos=b,                    
    keepspaces=true,                 
    numbers=left,                    
    numbersep=5pt,                  
    showspaces=false,                
    showstringspaces=false,
    showtabs=false,                  
    tabsize=2
}

\lstset{style=mystyle}



\newcommand{\what}[1] {\textcolor{red}{\textbf{#1} \addcontentsline{toc}{subsection}{\textcolor{orange}{#1}}}}

\newcommand{\todo}[1] {\textcolor{orange}{\textbf{TODO: #1}}}

\newcommand{\citeme}{\textcolor{orange}{\textbf{TODO: cite}}}

\newcommand{\phrasing}{\textcolor{green}{\textbf{phrasing}}}


\newcommand{\image}[3][1]
{
\begin{figure}
	\centerline{\includegraphics[width={1\linewidth}]{#1}}
	\caption{#2}
	\label{#3}
\end{figure}
}



\begin{document}

\title{\textbf{Philipps-Universität Marburg}}

\author{Johannes Gille}
\date{\parbox{\linewidth}{\centering%
    Fachbereich 17\endgraf
    AG Allgemeine und Biologische Psychologie\endgraf
    AE Theoretische Kognitionswissenschaft\endgraf
    \bigskip
    \bigskip
    Learning in cortical microcircuits with multi-compartment pyramidal neurons\endgraf
    \textbf{Supervisors:}\endgraf
    \bigskip
    Prof. Dr. Dominik Endres, Philipps-Universität Marburg\endgraf
    \bigskip
    Dr. Johan Kwisthout, Radboud University}}
\maketitle

\tableofcontents


\chapter{Introduction}



\section{Motivation}

The outstanding learning capabilities of the human brain have been found to be elusive and as yet impossible to fully
explain or replicate in silicio. While in recent years the power of classical machine learning solutions  has improved
even beyond human capabilities for some tasks, their underlying algorithms cannot serve as a model of human cognition.
Some reasons why brains and machines appear irreconcilable relate to questions about network structure and neuron
models. Yet more pressingly, almost all the most powerful artificial neural networks are trained with the
Backpropagation of errors algorithm, which has long been considered to be impossible for neurons to implement. Hence,
Neuroscience has dismissed this algorithm in an almost dogmatic way for many years after its development, stating that
the brain must employ a different mechanism to learn.

Yet in recent years, there has been a resurgence of research by neuroscientists towards reconciling biological and
artificial neural networks in spite of these concerns. This led to a number of experimental results indicating that
brains might be capable of performing something very similar to Backpropagation after all. Furthermore, despite rigorous
efforts, no unifying alternative to this learning principle was found which performs well enough to account for the
brain's unmatched capabilities.

Hence, there now exists a vibrant community developing alternative ways to implement this algorithm - or some
approximation of it. These novel approaches are capable of replicating an increasing number of properties of biological
brains. Nevertheless, many issues remain unsolved, and a lot of neuronal features remain unaccounted for in brain models
that are capable of any kind of learning. It is this open problem, to which I want to dedicate my efforts in this
thesis. After reviewing the existing literature, I have selected a promising model of learning in cortical circuits.
This model uses multi-compartment neuron models and local plasticity rules to implement a variant of Backpropagation. In
this project, I will investigate and attempt to further improve its concordance with data on the human neocortex. I will
use the approach of computationally modelling the model while progressively adding biological features, attempting to
retain learning performance in the process.


\section{The Backpropagation of errors algorithm}

The Backpropagation of errors algorithm (\textit{Backprop}) \citep{Schmidhuber2014} is the workhorse of modern machine
learning and is able to outperform humans on a growing number of tasks \citep{LeCun2015}. Particularly for training deep
neural networks it has remained popular and largely unchanged since its initial development. Its learning potential
stems from its unique capability to attribute errors in the output of a network to activations of specific neurons and
connections within its hidden layers. This property also forms the basis of the algorithm's name; After an initial
forward pass to form a prediction about the nature of a given input, a separate backward pass propagates the arising
error through all layers in reverse order. During this second network traversal, local error gradients dictate to what
extent a given weight needs to be altered so that the next presentation of the same sample would elicit a lower error in
the output layer. It has been argued that through this mechanism, Backprop solves the \textit{credit assignment problem}
- i.e.\ the question to what degree a parameter contributes to an error signal - optimally \citep{Lillicrap2020}. With
this critical information in hand, computing parameter changes that decrease error becomes almost trivial. As biological
neural networks are likewise subject to the credit assignment problem, finding a general solution to it promises to be
invaluable to neuroscience. For a long time Backprop was believed to be unsuitable for networks of biological neurons
for several reasons.


\section{Concerns over biological plausibility}

While Backprop continues to prove exceptionally useful in conventional machine learning systems, it is viewed critically
by many neuroscientists. For one, it relies on a slow adaptation of synaptic weights, and therefore requires a large
amount of examples to learn rather simple input-output mappings. In this particular way, its performance is far inferior
to the powerful one-shot learning exhibited by humans \citep{Brea2016}. Yet more importantly, no plausible mechanisms
have yet been found by which biological neural networks could implement the algorithm. In fact, Backprop as a way by
which brains may learn has been dismissed entirely by much of the neuroscience community for decades
\citep{Grossberg1987,Crick1989,Mazzoni1991,OReilly1996}. This dismissal is often focussed on three mechanisms that are
instrumental for the algorithm \citep{whittington2019theories,Bengio2015,Liao2016}:



\subsection{Local error representation}

Neuron-specific errors in Backprop are computed and propagated by a mechanism that is completely detached from the
network itself, which requires access to the entirety of the network state. In order to compute the weight changes for a
given layer, the algorithm takes as an input the activation and synaptic weights of all downstream neurons. In contrast,
plasticity in biological neurons is largely considered to be primarily dependent on factors that are local to the
synapse \citep{Abbott2000,magee2020synaptic,urbanczik2014learning}. While neuromodulators are known to influence
synaptic plasticity, their dispersion is too wide to communicate neuron-specific errors. Thus, biologically plausible
Backprop would require a method for encoding errors locally, i.e.\ close to the neurons to which they relate. This has
been perhaps the strongest criticism of Backprop in the brain, as many questions regarding mechanisms for both computing
and storing these errors remain unanswered as yet.

\subsection{The weight transport problem}

During the weight update stage of Backprop, errors are transmitted between layers with the same weights that are used in
the forward pass. In other words, the magnitude of a neuron-specific error that is back-propagated through a given
connection should be proportional to its impact on output loss during the forward pass. To replicate this, a neuronal
network implementing Backprop would require feedback connections that mirror both the precise connectivity and synaptic
weights of the forward connections. Bidirectional connections that could theoretically back-propagate errors are common
in the cortex, yet it is unclear by which mechanism pairs of synapses would be able to align. This issue becomes
particularly apparent when considering long-range pyramidal projections. In these, the feedforward and feedback synapses
which need to be aligned would potentially be separated by a considerable distance.

\subsection{Neuron models}

Finally, the types of artificial neurons typically used in Backprop transmit a continuous scalar activation at all
times, instead of discrete spikes. In theory, these activations correspond to the firing rate of a spiking neuron,
giving this class of models the title \textit{rate neurons}. Yet handling spike based communication requires more
sophisticated neuron models than are typically employed in Backprop networks. Additionally, plasticity rules for rate
neurons do not necessarily have an easily derived counterpart for spiking neurons. A notable example for this issue is
Backprop itself; The local error gradient of a neuron is not trivial to compute for spiking neural networks (SNN), as a
spiketrain has no natural derivative. Furthermore, a given neuron's activation in classical Backprop is computed from a
simple weighted sum of all inputs. This fails to capture the complex nonlinearities of dendritic integration that are
fundamental to cortical neurons (cf. Section \ref{sec-dendrites}). Finally, these abstract neurons - at least in
classical Backprop - have no persistence through time. Thus, their activation is dictated strictly by instantaneous
presynaptic activity, in contrast to the leaky membrane dynamics exhibited by biological neurons.


\section{Overcoming biological implausibility}

Backprop has remained the gold standard against which most attempts at modelling learning in the brain eventually are
compared. Also, despite its apparent biological implausibility, it does share some notable parallels to learning in the
brain. Artificial neural networks (ANN) trained with Backprop have been shown to develop similar representations to
those found in brain areas responsible for comparable tasks
\citep{Yamins2016,Whittington2018,KhalighRazavi2014,Kubilius2016}. Thus, numerous attempts have been made to define more
biologically plausible learning rules which approximate Backprop to some degree. A full review of the available
literature would be out of scope for this thesis, so only a few examples will be discussed in this section. \newline


\noindent One approach to solve the issues around local error representations is, to drive synaptic plasticity through a
global error signal \citep{potjans2011imperfect,mozafari2018combining,sutton2018reinforcement}. The appeal of this
solution is that such signalling could be plausibly performed by neuromodulators like dopamine
\citep{Mazzoni1991,Seung2003,izhikevich2007solving}. These types solutions to not approximate Backprop, but instead lead
to a kind of reinforcement learning. While some consider this the most plausible way for brains to learn
\citep{sutton2018reinforcement}, performance of global error/reward signalling stays far behind that of the credit
assignment performed by Backprop. Additionally, this class of algorithms requires even more examples of a training
dataset, and was shown to scale poorly with network size \citep{Werfel2003}. 

Two prominent classes of Backprop approximations have been developed, which are capable of locally representing errors.
These algorithms encode errors in either activation changes over time or local membrane potentials. They will be
discussed further in Section \ref{sec-model-selection}.\newline 


\noindent The weight transport problem was successfully addressed by a mechanism called \textit{Feedback Alignment} (FA)
\citep{Lillicrap2014}. This seminal paper shows that Backprop can still learn successfully when feedback weights are
random. In addition to learning to represent an input-output mapping in forward weights, the network is trained to
extract useful information from randomly weighted instructive pathways. The authors call this process \textit{learning
to learn}, and show that performance is even superior to classical Backprop for some tasks. This mechanism was further
expanded to show that the principles of FA perform very well when biologically plausible plasticity rules are employed
\citep{Liao2016,Zenke2018}. Another popular line of thought is - instead of computing local errors - to compute optimal
activations for hidden layer neurons using autoencoders \citep{Bengio2014,Lee2015,Ahmad2020}. Approaches derived from
this do not suffer from the weight transport problem, and by design does not require local error representations. While
these solutions (summized as \textit{Target propagation} algorithms) solve the weight transport problem, they fall far
behind traditional Backprop on more complex benchmark datasets like
\textit{\href{https://www.cs.toronto.edu/~kriz/cifar.html}{CIFAR}} and
\textit{\href{https://www.image-net.org/index.php}{ImageNet}} \citep{Bartunov2018}.\newline

\noindent Numerous approaches for implementing Backprop with more plausible neuron models exist, most of which employ variants of
the \textit{Leaky Integrate-and-fire} (LIF) neuron \citep{Sporea2013,Lee2016,Bengio2017,Lee2020}. The aforementioned
issue of computing the derivative over spiketrains has been solved in several ways, with the most prominent variant
perhaps being \textit{SuperSpike} \citep{Zenke2018}. One might therefore view this as the weakest criticism aimed at
Backprop. Yet none of the employed neuron models come close to portraying the intricacies of biological neurons, and
thus fail to provide explanations for their complexity. One aspect of this will be discussed in the upcoming
section.\newline

\noindent All of these studies successfully solve one or more concerns of biological plausibility, while still
approximating Backprop to some degree. Yet none of them are able to solve all three simultaneously, and some of them
introduce novel mechanisms that are themselves biologically questionable. It further appears that in all but a few
cases, an increase in biological plausibility leads to a decrease in performance. Thus, whether Backprop could be
implemented or approximate by biological neurons remains an open question.

\subsection{Dendrites as computational elements}\label{sec-dendrites}

The issue of oversimplified neuron models is by far the most frequent to be omitted from explanations of the biological
implausibility of Backprop (See for example \citep{Meulemans2020,Lillicrap2014}). This disregard might stem from the
fact that rate-based point neurons are employed in many of the most powerful artificial neural networks. This fact might
be taken as an argument that the simple summation of synaptic inputs is sufficient for powerful and generalized
learning. Modelling neurons more closely to biology would by this view only increase mathematical complexity and
computational cost without practical benefit. Another hypothesis states that the dominance of point neurons stems from a
"somato-centric perspective" within neuroscience \citep{Larkum2018}, which stems from the technical challenges inherent
to studying dendrites in vivo. The vastly different amount of available data regarding these two neuronal components
might have induced a bias in how neurons are modelled computationally. Some researchers have even questioned whether
dendrites should be seen as more of a 'bug' than a 'feature' \citep{Haeusser2003}, i.e.\ a biological necessity which
needs to be overcome and compensated for.

Yet in recent years, with novel mechanisms of dendritic computation being discovered, interest in researching and
explicitly modelling dendrites has increased. Particularly the vast dendritic branches of pyramidal neurons found in the
cerebral cortex, hippocampus and amygdala, were shown to perform complex integrations of their synaptic inputs
\citep{spruston2008pyramidal}. These dendritic trees are capable of performing coincidence- \citep{Larkum1999} and
sequence detection \citep{Branco2010}. The size of dendritic trees is also known to discriminate regular spiking from
burst firing pyramidal neurons \citep{Elburg2010}. Furthermore, pyramidal neuron dendrites are capable of performing
computations, which were previously assumed to require multi-layer neural networks \citep{Schiess2016,Gidon2020}. See
\citep{Larkum2022} and \citep{Poirazi2020} for extensive reviews. 

These neuroscientific insights have  sparked hope that modelling dendritic compartments explicitly might aid machine
learning in terms in both learning performance and energy efficiency
\citep{Chavlis2021,guerguiev2017towards,Richards2019,Eyal2018}. It appears that, if not for computational gains,
dendrites should be considered essential for any model attempting to explain the power of human learning. While the
network discussed in this thesis includes very simple multi-compartment models, the choice of model was
strongly influenced by the fact that segregated dendrites were considered at all.




\section{Cortical microcircuits}

Another feature of the brain which is often not considered in (biologically plausible) machine learning models is its
intricate connectivity. This is quite understandable, as there is still some uncertainty about which brain areas would
be involved in Backprop-like learning. It is also unclear, to what level of detail these areas would need to be modeled.
It has been shown that the connectivity patterns of cortical circuits are superior to amorphous networks in some cases
\citep{haeusler2007statistical}, so there might be a computational gain from modeling network structure closer to
biology. The question over network structure goes hand in hand with the choice of neuron models, as synaptic connections
arrive at specific points of pyramidal neuron dendrites, depending on the origin of the connection
\citep{felleman1991distributed,Ishizuka1995,Larkum2018}.

Several theories of cortical funcition focus more on reinforcement \citep{Legenstein2008} or unsupervised learning
\citep{George2009,hausler2017inhibitory}. Without dismissing these theories, this thesis will adopt the viewpoint that
human brains require a form of gradient descent to successfully adapt to their ever-changing environments. Furthermore,
we share the hypothesis that this kind of learning occurs predominantly in the neocortex \citep{Marblestone2016}.

The literature on the subject of learning historically appears to be somewhat split (although several important
exceptions have been published recently). On the one hand, the "machine-learning" point of view largely considers the
utility of network changes first, with considerations of biology appearing as an afterthought\citeme. On the other hand,
intricate models of cortical circuits exist, which can so far not be trained to perform tasks
\cite{potjans2014cell,schmidt2018multi,van2022bringing}. Within this thesis, I hope to contribute to the body of
literature between those extremes. For this, my approach will be to select a learning model that is already highly
biologically plausible, and to attempt to improve its plausibility - without breaking the learning rule.

\section{Model selection}\label{sec-model-selection}

The model selection progress was strongly influenced by a review article on biologically plausible approximations of
Backprop \citep{whittington2019theories}. The authors narrow the wide range of proposed solutions down to four
algorithms that are both highly performant and largely biologically plausible. Due to impact of the paper on this
thesis, their model comparison is depicted in Supplementary Table \ref{tab-wb-models}. The algorithms were in part
selected for requiring minimal external control during training, as well as by the fact that they can all be described
within a common framework of energy minimization \citep{Scellier2017}. The first two models are Contrastive learning
\citep{OReilly1996}, and its extension to time-continuous updates \citep{Bengio2017}. Both of these encode
neuron-specific errors in the change of neural activity over time. One of their appeals is the fact that they rely on
Hebbian (and Anti-Hebbian) plasticity, which are highly regarded in the neuroscience literature
\citep{magee2020synaptic,Brea2016}. Yet in the plasticity rule also lies their greatest weakness, as synapses need to
switch between the two opposing mechanisms once the target for a given stimulus is provided. This switch requires a
global signal that communicates the change in state to all neurons in the network simultaneously.

The second class of models was more appealing to me, as both variants are based on the predictive coding account in
Neuroscience \citep{rao1999predictive}, which deserves its own introduction.

\subsection{Predictive coding}

In this seminal model of processing in the visual cortex, each level of the visual hierarchy represents the outside
world at some level of abstraction. Recurrent connections then serve communicate prediction errors and predictions up
and down the hierarchy respectively, which the network attempts to reconcile. The authors show that through rather
simple computations, these prediction errors can be minimized to obtain useful representations at each level of the
hierarchy. They further show that a predictive coding network trained on natural images exhibits end-stopping properties
previously found in mammalian visual cortex neurons. This work was instrumental to shaping the modern neuroscientific
perspective of perception being largely driven by cortico-cortical feedback connections in addition to the feedforward
processes. The extension of predictive coding principles from visual processing to the entire living system is promising
to revolutionize neuroscience under the name of \textit{Active inference} \citep{Friston2008,Friston2009,Adams2015}. By
this view, the entire brain aims to minimize prediction errors with respect to an internal (generative) model of the
world. A noteworthy property of this hypothesis is that it implies an agents action in the world as 'just another' way
in which it can decrease discrepancies between its beliefs and sensory information. In a seminal paper, a model of the
cortical microcircuit \citep{haeusler2007statistical} was shown to have a plausible way for performing the computations
required by predictive coding \citep{bastos2012canonical}.

While predictive coding was originally described as a mechanism for unsupervised learning, through a slight modification
it is also capable of performing Backprop-like supervised learning \citep{Whittington2017}. This is the third model
considered in the review paper, in which values (i.e.\ predictions) and errors of a layer are encoded in separate,
recurrently connected neuron populations. By employing only local Hebbian plasticity, this network is capable of
approximating Backprop in multilayer perceptrons while conforming to the principles of predictive coding. The constraint
on network topology was later relaxed by showing that the model is capable of approximating Backprop for arbitrary
computation graphs \citep{Millidge2022}. The neuron-based predictive coding network was therefore an important
contribution towards unifying the fields of Active inference and machine learning research. As noted in a recent review
article:

\begin{quotation}
  \noindent``Since predictive coding is largely biologically plausible, and has many potentially plausible process
  theories, this close link between the theories provides a potential route to the development of a biologically
  plausible alternative to backprop, which may be implemented in the brain. Additionally, since predictive coding can be
  derived as a variational inference algorithm, it also provides a close and fascinating link between backpropagation of
  error and variational inference.`` \citep{millidge2021predictive} \end{quotation}

\noindent With this perspective in mind, we turn to the final model discussed in the review paper.

\subsection{The Dendritic error model}

The predictive coding network stores local prediction errors in nodes (i.e.\ neurons) close to the nodes to which these
errors relate. That errors may be represented within the activation of individual neurons is a promising hypothesis with
some advantages, as well as results backing it up \citep{Hertaeg2022}. Yet there is a competing view, by which errors
elicited by individual neurons may be represented by membrane potentials of their dendritic compartments
\citep{guerguiev2017towards}. The "Dendritic error model" \citep{sacramento2018dendritic} - as the name implies -
follows this line of thought. It contains a highly recurrent network of both pyramidal- and interneurons, in which
pyramidal neuron apical dendrites encode prediction errors. This view is supported by behavioral rodent experiments
which show that stimulation to pyramidal neuron apical tufts in cortical layer 1 controls learning \citep{Doron2020}.

For the errors to be encoded successfully, the model requires a symmetry between feedforward and feedback sets of
weights, which it has to learn prior to training. After that, apical compartments behave like the error nodes in a
predictive coding network. They are silent during a feedforward network pass, and encode local prediction errors in
their membrane potential when a target is applied to the output layer. Since they are a part of the pyramidal neuron,
only local information is required to minimize these prediction errors through a plasticity rule for multi-compartment
neurons \citep{urbanczik2014learning}. A critical observation made in \citep{whittington2019theories} is that the
dendritic error model is mathematically equivalent to their predictive coding network \todo{expand if I have time,
otherwise this will be a ref.}. All of these factors combined make the dendritic error model a promising model to help
us further understand both predictive coding and deep learning in cortical circuits. While both the employed neuron and
connectivity model are far behind some of the more rigorous cortical simulations, it is regarded in the
literature as an important step towards integrating deep learning and neuroscience.

Nevertheless, the model still suffers from some constraints with regard to its biological plausibility; Both the
predictive coding network and the dendritic error network require strongly constrained connectivity schemes, without
which they cannot learn. This kind of specificity (in particular one-to-one relationships between pairs of neurons) are
highly untypical for cortical connections \citep{Thomson2003}. Hence, their exact network architectures are unlikely to
be present in the cortex. The Dendritic error model additionally requires Pre-training to be capable of approximating
Backprop. Both of these issues will be discussed in this thesis. Yet the most salient improvement to the network's
biological plausibility is likely, to change neuron models from rate-based to spiking neurons. It has been shown that
the Plasticity rule employed by the network is capable of performing simple learning tasks when adapted to spiking
neurons \citep{Stapmanns2021}. Yet, (to the best of my knowledge) there are no studies investigating if this variant is
capable of learning more complex tasks on a network-level. A spiking implementation of the dendritic error network will
therefore be the starting point for this thesis, upon which further analysis shall build.



\chapter{Methods}

\todo{from \cite{Haider2021}: To differentiate between biologically plausible, leaky neurons and abstract neurons with instantaneous
  response, we respectively use the terms “neuronal” and “neural”.}

\section{The Backpropagation of errors algorithm}

The Backpropagation of errors algorithm (henceforth referred as "Backprop") forms the backbone of modern machine
learning. \citeme It is as of yet unmatched with regard to training in deep neural networks due to its unique capability to
attribute errors in the output of a network to activations of specific neurons within its hidden layers and adapt 
incoming weights in order to improve network performance. This property also forms the basis of the algorithm's name;
After an initial forward pass to form a prediction about the nature of a given input, a separate backward pass 
propagates the arising error through all layers in reverse order. During this second network traversal, local error
gradients dictate, to what extent a given weight needs to be altered so that the next presentation of the same sample
would elicit a lower error in the output layer. 


While Backprop continues to prove exceptionally useful in conventional machine learning systems, attempts use it to
explain the exceptional learning capabilities of the human brain have so far not been successful \phrasing. In fact,
Backprop as a mechanism for synaptic plasticity in the brain is dismissed by many neuroscientists as biologically 
implausible. This dismissal is primarily based on three mechanisms that are instrumental for Backprop
\cite{whittington2019theories}:

\subsection*{The backward pass}


\subsection*{Local error representation}

\subsection*{The weight transport problem}

Yet, despite these mechanisms being at odds with biology, when training on real-world data, artificial neural networks
 have been shown to learn similar representations as those found in brain areas responsible for comparable tasks
 \cite{whittington2019theories,Yamins2016}. 


\todo{discuss supervisor issue?}


\section{Neuron and network model}

\begin{figure}
  \centerline{\includegraphics[width={1\linewidth}]{pyramidal.png}}
  \caption{Self-predicting initialization without plasticity}
\end{figure}
\section{Urbanczik-Senn Plasticity}


from \cite{Haider2021}:

\textit{In this architecture, plasticity serves two purposes. For pyramidal-to-pyramidal feedforward synapses,
  it implements error-correcting learning as a time-continuous approximation of BP. For pyramidal-to-
  interneuron synapses, it drives interneurons to mimic their pyramidal partners in the layers above (see
  also SI). Thus, in a well-trained network, apical compartments of pyramidal cells are at rest, reflecting
  zero error, as top-down and lateral inputs cancel out. When an output error propagates through the
  network, these two inputs can no longer cancel out and their difference represents the local error ei .
  This architecture does not rely on the transpose of the forward weight matrix, improving viability for
  implementation in distributed asynchronous systems. Here, we keep feedback weights fixed, realizing
  a variant of feedback alignment. In principle, these weights could also be learned in order to further
  improve the local representation of errors Section 7.}




One-sided exponential decay kernel

\begin{align}
  \kappa(t) & = H(t)e^{-t/\tau_{\kappa}} \\
  H(t)      & =
  \begin{cases}
    1 & \text{if $t > 0$}    \\
    0 & \text{if $t \leq 0$} \\
  \end{cases}
\end{align}

Antiderivatives:

\begin{align}
  \int_{-\infty}^x H(t)dt = tH(t) = max(0,t)
\end{align}

Convolution:

\begin{align*}
  (f \ast g)(t) & = \int_{- \infty }^{\infty} f(\tau) g(t-\tau) d \tau
\end{align*}
For functions f, g supported on only $[0, \infty]$ (as one-sided decay kernels and spiketrains are), integration limits can be truncated:
\begin{align*}
  (f \ast g)(t) & = \int_{0}^{t} f(\tau) g(t-\tau) d \tau \\
\end{align*}


Plasticity:

\begin{align}
  \frac{dW_{ij}}{dt}(t) & = F(W_{ij}(t), s_i^\ast (t), s_j^\ast (t), V_i^\ast (t)) \\
  F[s_j^\ast, V_i^\ast] & = \eta \kappa \ast (V_i^\ast s_j^\ast)                   \\
  \text{with } V_i^\ast & = (s_i - \phi(V_i )) h(V_i),                             \\
  s_j^\ast              & = \kappa_s \ast s_j.
\end{align}

For an event-based plasticity we need:

\begin{align}
  \Delta W_{ij}(t,T) & = \int_t^T dt' F[s_j^\ast , V_i^\ast ](t')                                                 \\
                     & = \int_t^T dt' \eta \kappa \ast (V_i^\ast s_j^\ast)                                        \\
                     & = \eta \int_t^T dt' \  \int_0^{t'} dt'' \ \kappa(t'-t'') V_i^\ast (t'') s_j^\ast (t'')     \\
                     & = \eta \int_0^t dt' \  \int_{t''}^{t'} dt'' \ \kappa(t'-t'') V_i^\ast (t'') s_j^\ast (t'') \\
\end{align}


Starting with the complete Integral from $t=0$.

\begin{align*}
  \Delta W_{ij}(0,t) & =\eta \int_0^t dt' \  \int_0^{t'} dt'' \ \kappa(t'-t'') V_i^\ast (t'') s_j^\ast (t'')                          \\
                     & = \eta \int_0^t dt'' \  \int_{t''}^{t} dt' \ \kappa(t'-t'') V_i^\ast (t'') s_j^\ast (t'')                      \\
                     & = \eta \int_0^t dt'' \  \left[ \tilde{\kappa}(t-t'') - \tilde{\kappa}(0) \right] V_i^\ast (t'') s_j^\ast (t'') \\
\end{align*}

With $\tilde{\kappa}$ being the antiderivative of $\kappa$:

\begin{align*}
  \kappa(t)         & = \frac{\delta}{\delta t} \tilde{\kappa}(t) \\
  \tilde{\kappa}(t) & = - e^{-\frac{t}{t_{\kappa}}}               \\
\end{align*}

The above can be split up into two separate integrals:
\begin{align*}
  \Delta W_{ij}(0,t) & =\eta \left[ -I_2 (0, t) + I_1(0,t) \right]                                      \\
  I_1(t_1, t_2)      & = - \int_{t_1}^{t_2} dt' \ \tilde{\kappa} (0) V_i^\ast (t') s_j^\ast (t')        \\
  I_2(t_1, t_2)      & = - \int_{t_1}^{t_2} dt' \ \tilde{\kappa} (t_2 - t') V_i^\ast (t') s_j^\ast (t') \\
\end{align*}

Which implies the identities

\begin{align*}
  I_1(t_1, t_2 + \Delta t) & = I_1 (t_1, t_2) + I_1 (t_2, t_2 + \Delta t)                                       \\
  I_2(t_1, t_2 + \Delta t) & = e^{- \frac{t_2 - t_1}{\tau_{\kappa}}} I_2 (t_1, t_2) + I_2 (t_2, t_2 + \Delta t)
\end{align*}


\begin{align}
  I_2 (t_1, t_2 + \Delta t) & = -\int_{t_1}^{t_2 + \Delta t} dt' \ \tilde{\kappa} (t_2 + \Delta t - t') V_i^\ast (t') s_j^\ast (t')                                        \\
                            & = -\int_{t_1}^{t_2} dt' \ \left[ -e^{- \frac{t_2 + \Delta t - t'}{\tau_\kappa}} \right] V_i^\ast (t') s_j^\ast (t')
  -\int_{t_2}^{t_2 + \Delta t} dt' \ \left[ -e^{- \frac{t_2 + \Delta t - t'}{\tau_\kappa}} \right] V_i^\ast (t') s_j^\ast (t')                                             \\
                            & = -e^{- \frac{ \Delta t}{\tau_\kappa}} \int_{t_1}^{t_2} dt' \ \left[ -e^{- \frac{t_2 - t'}{\tau_\kappa}} \right] V_i^\ast (t') s_j^\ast (t')
  -\int_{t_2}^{t_2 + \Delta t} dt' \ \left[ -e^{- \frac{t_2 + \Delta t - t'}{\tau_\kappa}} \right] V_i^\ast (t') s_j^\ast (t')
\end{align}


Using this we can rewrite the weight change from $t$ to $T$ as:


\begin{align*}
  \Delta W_{ij}(t,T) & = \Delta W_{ij}(0,T) - \Delta W_{ij}(0,t)                                               \\
                     & = \eta [-I_2(0,T) + I_1(0,T) + I_2(0,t) - I_1(0,t)]                                     \\
                     & = \eta [I_1(t,T) - I_2(t,T) + I_2(0,t)\left( 1 - e^{- \frac{T-t}{\tau_\kappa}} \right)]
\end{align*}

The simplified \cite{sacramento2018dendritic} case would be:

\begin{align*}
  \frac{dW_{ij}}{dt} & = \eta (\phi(u_i) - \phi(\hat{v_i})) \phi(u_j)                                         \\
  \Delta W_{ij}(t,T) & = \int_t^T dt' \ \eta \  (\phi(u_i^{t'}) - \phi(\widehat{v_i^{t'}})) \  \phi(u_j^{t'}) \\
  \Delta W_{ij}(t,T) & = \eta \int_t^T dt' \  (\phi(u_i^{t'}) - \phi(\widehat{v_i^{t'}})) \ \phi(u_j^{t'})    \\
  V_i^*              & = \phi(u_i^{t'}) - \phi(\widehat{v_i^{t'}})                                            \\
  s_j^*              & = \kappa_s * s_j
\end{align*}


Where $s_i$ is the postsynaptic spiketrain and $V_i^*$ is the error between dendritic prediction and somatic rate and $h( u )$. The additional nonlinearity $h( u ) = \frac{d}{du} ln \  \phi(u)$ is ommited in our model \todo{should it though?}.



\begin{align}
  \tau_l & = \frac{C_m}{g_L} = 10 \\
  \tau_s & = 3
\end{align}

Writing membrane potential to history (happens at every update step of the postsynaptic neuron):

\begin{lstlisting}[language=C++, directivestyle={\color{black}}
                   emph={int,char,double,float,unsigned,exp},
                   emphstyle={\color{blue}}]

UrbanczikArchivingNode< urbanczik_parameters >::write_urbanczik_history(Time t, double V_W, int n_spikes, int comp)
{
	double V_W_star = ( ( E_L * g_L + V_W * g_D ) / ( g_D + g_L ) );
	double dPI = ( n_spikes - phi( V_W_star ) * Time::get_resolution().get_ms() )
      * h( V_W_star );
}\end{lstlisting}

I interpret this as:


\begin{align*}
  \int_{t_{ls}}^T dt' \ V_i^* & = \int_{t_{ls}}^T dt' \  (s_i - \phi(V_i )) h(V_i),               \\
  \int_{t_{ls}}^T dt' \ V_i^* & = \sum_{t=t_{ls}}^T \  (s_i(t) -  \phi(V_i^t ) \Delta t) h(V_i^t) \\
\end{align*}

\begin{lstlisting}[language=C++, directivestyle={\color{black}}
                   emph={int,char,double,float,unsigned,exp},
                   emphstyle={\color{blue}}]
for (t = t_ls; t< T; t = t + delta_t)
{
   	minus_delta_t = t_ls - t;
    minus_t_down = t - T;
    PI = ( kappa_l * exp( minus_delta_t / tau_L ) - kappa_s * exp( minus_delta_t / tau_s ) ) * V_star(t);
    PI_integral_ += PI;
    dPI_exp_integral += exp( minus_t_down / tau_Delta_ ) * PI;
}  
// I_2 (t,T) = I_2(0,t) * exp(-(T-t)/tau) + I_2(t,T)
PI_exp_integral_ = (exp((t_ls-T)/tau_Delta_) * PI_exp_integral_ + dPI_exp_integral);
W_ji = PI_integral_ - PI_exp_integral_;
W_ji = init_weight_ + W_ji * 15.0 * C_m * tau_s * eta_ / ( g_L * ( tau_L - tau_s ) );    
  
kappa_l = kappa_l * exp((t_ls - T)/tau_L) + 1.0;
kappa_s = kappa_s * exp((t_ls - T)/tau_s) + 1.0;
  \end{lstlisting}


\begin{align*}
  \int_{t_{ls}}^T dt' s_j^* & =  \tilde{\kappa_L}(t') * s_j -  \tilde{\kappa_s}(t') * s_j
\end{align*}

$I_1$ in the code is computed as a sum:

\begin{align}
  I_1 (t,T) = \sum_{t'=t}^T \ (s_L^*(t') - s_s^*(t')) * V^*(t')
\end{align}


\section{steady-state potentials in Sacramento (2018)}

\begin{align*}
  u_k^p           & = \frac{g_B}{g_{lk} + g_B + g_A} v^P_{B,k} + \frac{g_A}{g_{lk} + g_B + g_A} v^P_{A,k} \\
  \hat{v}^P_{B,k} & = \frac{g_B}{g_{lk} + g_B + g_A} v^P_{B,k}                                            \\
  \hat{v}^I_{k}   & = \frac{g_B}{g_{lk} + g_B} v^I_{k}                                                    \\
  \lambda         & = \frac{g_{som}}{g_{lk} + g_B + g_{som}}
\end{align*}






\section{Multi-compartment neuron models}



\section{Cortical microcircuits}

\section{Latent equilibrium}

\begin{align*}
  \phi(V^{som}) & \rightarrow \phi(\breve{V}^{som}) \\
  \breve{V}     & := V + \tau^m \dot{V}             \\
\end{align*}
\begin{align*}
  \frac{d}{dt} W_{ba} & = \eta (\phi(V_b^{som}) - \phi(\alpha V_b^{dend})) \phi(V_a^{som})                         \\
  \frac{d}{dt} W_{ba} & = \eta (\phi(\breve{V}_b^{som}) - \phi(\alpha \breve{V}_b^{dend})) \phi(\breve{V}_a^{som})
\end{align*}


\section{simulation details/updates}

All voltages need to be reset between simulations!



\chapter{Results}

The following results are exploratory in nature, and After some poor initial results the focus was laid on proving that
the network can perform at all, rather than fine-tuning hyperparameters towards optimal performance. This decision was
in part motivated by a priorization of gaining neuroscientific insights over targeting minimal loss. It should be noted,
that training the network is computationally quite costly (c.f. Section \ref{sec-benchmark}) which turned parameter
studies into a time-consuming process.

Early experiments showed that the network is rather sensitive to parameter changes. The search for default parameters
took some effort, as a certain heterogeneity exists in the two existing implementations
\citep{sacramento2018dendritic,Haider2021}, both in hyperparameters as in the simulation environment. This model
includes properties of both variants, while relying more strongly on the LE implementation. Unless stated otherwise,
neurons employ prospective activation functions in all simulations. So far, no drawbacks to this mechanism have
presented themseleves, and learning speed can be increased drastically compared to the original implementation. The full
default parametrization is shown in Supplementary Table \ref{tab-params}. Since it was anticipated that the spiking
implementation would perform worse than the rate-based variant, the first goal was to measure how big this difference in
performance is. Furthermore, a relevant question was to what degree the synaptic delays enforced by NEST would influence
performance of the rate model. These questions will be answered in the upcoming sections. Note that not all experimental
results were given their own Figures. In these cases, plots can be found in the electronic supplementary material.


\section{The self-predicting state}

As a first comparison between the three implmementations, the pre-training towards a self-predicting state (cf.
\citep{sacramento2018dendritic}[Fig. S1]) was performed. For this experiment, no target signal is provided at the output
layer, and the network is tasked with learning to self-predict top-down input. The network is initialized with fully
random weights and stimulated with random inputs from a uniform distribution between 0 and 1. A comparison of the four
error metrics between implementations is shown in Fig. \ref{fig-self-pred}.



\begin{figure}[h]
    \centering
    \includegraphics[width=0.9\textwidth]{fig_self_prediction}
    \caption[Training towards the self-predicting state]{Training towards the self-predicting state. All implementations
        learn to predict self-generated top-down signals. Networks were initialized with the same random weights for
        dimensions $[5, 8, 3]$, and stimulated with $5000$ samples of random input for $100ms$ each. As described in
        \citep{sacramento2018dendritic}, during this phase only $Pyr \rightarrow Intn$ and $Intn \rightarrow Pyr$
        weights are plastic ($\eta^{pi}=0.05, \eta^{ip}=0.02375, \eta^{up}_0=\eta^{up}_1=\eta^{down}=0$).}
    \label{fig-self-pred}
\end{figure}

Both rate neuron implementations were able to reach comparable values for all error metrics after roughly the same time.
The exact values that errors converge on differs slightly between implementations, with no implementation being clearly
superior. This is an important result for upcoming experiments, as it indicates that both training environment (current
injections, simulation time, membrane reset, readouts, etc.) and the actual neuron model of the NEST version adequately
replicate the original model.

For the spiking variant, Interneuron- and its corresponding feedfworward weight error are comparable to the other
implementations. In fact these metrics appear to converge slightly faster to comparable values. The primary limitation
of this version are the apical error and the closely correlated feedback weight error. After appearing to converge very
quickly, the two metrics stagnate at very high levels. These high errors correlate with strong fluctuations of the
apical compartment. These fluctuations can likely at least in part be attributed to low spike frequencies. This was
confirmed by repeating the experiment with $\psi=1500$, which alleviated the issue to a degree (results not shown). Yet,
error values were still inferior to the rate models, and this change came at the cost of substantially increased
training time. Therefore, this approach was not pursued much further. A different possible solution is, to increase the
membrane capacitance of the apical compartment in order to smoothe out the fluctuations induced by individual spikes.
This will be discussed in Section \ref{sec-c-m-api}.

In most simulations in the literature, the network is initialized to an ideal self-predicting state. Furthermore
feedback weights are non-plastic in many experiments ($\eta^{pi}=\eta^{down}=0$). Therefore, a failure to perfectly
learn this weight symmetry should not fundamentally hinder learning. For the time being, showing that the network
approaches a self-predicting state was deemed a sufficient result.


\section{Presentation times and latent equilibrium}\label{sec-le-tpres}

In order to validate the performance of the NEST implementations on a learning task, the parameter study from
\citep{Haider2021}[Fig. 3] was replicated. In this experiment, the network is trained with different stimulus
presentation times $t_{pres} \in \{0.3,\ 500\}ms$. Performance of the original Dendritic error network is compared to
the improved model which employs LE. Due to the costly computation of the network under such long $t_{pres}$, a simple
artificial classification dataset was used. The \textit{Bars-dataset} is defined for $3\times3$ input- and $3$ output
neurons. it consists of three horizontal, three vertical, and two diagonal bars in the $3\times3$ grid, which are to be
encoded in a 'one-hot-vector' at the output layer. In the experiment, networks of $9-30-3$ pyramidal neurons per layer
were trained for 1000 Epochs of 24 samples each. Networks were initialized to the self-predicting state and only
feedforward $Pyr\rightarrow Pyr$ and $Pyr \rightarrow Intn$ synapses were plastic. Learning rates scaled inversely with
presentation times: $\eta^{ip}_0 = \frac{0.2}{t_{pres}}, \eta^{up}_0 = \frac{0.5}{t_{pres}}, \eta^{up}_1 =
    \frac{0.1}{t_{pres}}$. The results for the spiking NEST network are shown in Fig. \ref{fig-bars-le-snest}, while the
results for NumPy and Rate NEST variants are depicted in Supplementary Figures \ref{fig-bars-le-numpy} and
\ref{fig-bars-le-rnest}, respectively.


\begin{figure}[h]
    \centering
    \includegraphics[width=0.9\textwidth]{fig_3_snest}
    \caption[Replication of Fig. 3 from \citep{Haider2021}]{Replication of Fig. 3 from \citep{Haider2021} using networks
        of spiking neurons in the NEST simulator. \textbf{A:} Comparison between Original dendritic error network by and
        an identical network employing Latent equilibrium. Shown is the training of networks with 9-30-3 neurons on the
        Bars-dataset from with three different stimulus presentation times. \textbf{B:} Test performance after 1000
        Epochs as a function of stimulus presentation time.}
    \label{fig-bars-le-snest}
\end{figure}

For the original dendritic error model, performance in all three implementations is close to being identical. This an
important finding as it answers two open questions: Changes made for a NEST-compatible implementation were adequate and
result in identical learning behaviour between the rate-based implementations. Learning behaviour of the spiking model
is competetive, confirming the hypothesis that the spike-based dendritic plasticity model is capable of more complex
credit assignment tasks than previously shown. In this regard, the implementation can be considered a success.

The results for the LE network experiments are somewhat more interesting. For very long $t_{pres}$, both rate
implementations behave the same. yet the NEST implementation requires considerably more epochs for training, as
$t_{pres}$ is reduced. For very low presentation times, this behavior was somewhat expected, due to the synaptic delay
enforced by NEST. The NumPy variant computes a full forward pass of the network during a single simulation step, as all
layers are processed in sequence. Only feedback signals from pyramidal neurons are delayed by one timestep in this
simulation backend. In NEST, all connections have a minimum synaptic delay of $\Delta t$. Therefore, for very short
presentation times the NEST network can not be expected to perform well, as signals have no time to traverse the
network. It remains an open question whether this feature alone explains the gradual decrease in performance observed
here, or if there is an undiscovered error within the novel neuron models or simulation environment. The exceptionally
short stimulus presentation times investigated by \citep{Haider2021} are themselves questionable in terms of biological
plausibility, as they are much lower than pyramidal neuron time constants \citep{McCormick1985}. Thus, no attempts were
made to improve performance for very low $t_{pres}$.

The spiking variant proved similarly sensitive to presentation times as the other NEST variant. While obtaining similar
final accuracy, it required - at best - twice as many stimulus presentations as its direct competitor. This result,
while somewhat expected, shows that for low $t_{pres}$, spiking communication leads to worse learning performance. It
also shows that the relaxation problem affects all communication schemes equally. While the utility of LE is
substantially higher for rate neurons, it does improve performance and efficiency of the spiking variant. For this
reason, LE will be turned on in all upcoming simulations.

\section{Approximating arbitrary functions}\label{sec-func-approx}

To confirm that the spiking network is capable of learning more complex tasks, it was trained to match the input-output
mapping of a separate teacher network. This is an established method for showing that a network can approximate
arbitrary functions. A comparison between different 

\begin{figure}[h]
    \centering
    \includegraphics[width=0.8\textwidth]{fig_function_approximator}
    \caption[SNN learns to match separate teacher network]{SNN learns to match separate teacher network. Networks of
    size $15-n_{hidden}-5$ neurons per layer were trained on input-output pairings of a randomly initialized feedforward
    network of size $15-15-5$. Each network was trained on 300000 samples of the same teacher network while all somatic
    compartments received background noise with a standard deviation of $0.01$.}
    \label{fig-func-approx}
\end{figure}

As the task was too easy when inputs to the teacher network were strictly positive (as is the case in the
self-prediction paradigm), inputs were drawn from a uniform distribution $U(-1,1)$. To facilitate inhibitory stimulation
to the spiking network, a separate population was required at the input layer. This population was initialized with
inverted weights compared to the excitatory population and received all negative values of a given input vector. Due to
this necessity, the spiking network had to learn an additional set of weights, wich might be one of the reasons why
overall performance stayed behind expectations.

\todo{complete}

\section{Apical compartment capacitance}\label{sec-c-m-api}

Next, an investigation was made into the apical- and feedback weight errors of the SNN. The hypothesis was that a
smoothing of apical compartment voltage would lead to a decrease in both errors. To test this, the Self-predicting
experiment was repeated with numerous values for apical compartment capacitance $C_m^{api} \in \{ 1, 250 \} pF$ (results
not shown). The experiment was a success and showed that for $C_m^{api} = 50pF$, apical error is almost halved ($0.0034
    \rightarrow 0.0019$), and FB error is decreased by $80\%$ ($0.15 \rightarrow 0.027$). These values are still at least an
order of magnitude higher than the rate implementations, but mark a substantial improvement. Further increasing the
parameter lead to a decreased apical error, but came at the cost of slower convergence. Higher membrane capacitances
must in general be expected to increase the relaxation period of the entire network. Thus, they requiring a highly
undesirable increase in $t_{pres}$ for successful learning.

\begin{figure}[h]
    \centering
    \includegraphics[width=0.9\textwidth]{fig_c_m_psi}
    \caption[Comparison of performance for different configurations of $\psi$ and $C_m^{api}$]{Comparison of performance
    for different  configurations of $\psi$ and $C_m^{api}$.}
    \label{fig-c-m-psi}
\end{figure}

A secondary objective of experimenting with apical capacitance was to enable learning with lower $\psi$ and therefore
approach biologically plausible firing rates. To test this, training on the Bars dataset was performed under different
combinations of apical capacitance and $\psi$. \todo{describe params and results}



Hence, another tradeoff between performance and training duration is introduced by this parameter.


\section{Imperfect connectivity}

Connectivity within cortical circuits, while structured, appears to subject to a high degree of randomness
\citep{potjans2014cell} As noted before, one-to-one connections between pairs of neurons are therefore highly unlikely
\citep{whittington2019theories}. On the other hand, 'fully connected' populations of neurons likewise have not been
observed in electrophysiological \citep{thomson2002synaptic} and in-vitro \citep{binzegger2004quantitative} analyses of
cortico-cortical connectivity. Therefore, any network proclaiming to model the cortex must invariably be capable of
handling this imperfect connectivity.

\begin{figure}[h]
    \centering
    \includegraphics[width=0.9\textwidth]{fig_dropout}
    \caption[Error terms after training with synapse dropout]{Error terms after training with synapse dropout. Networks
        with $8-8-8$ neurons per layer were trained towards the self-predicting state, with different percentages of
        synaptic connections randomly removed. Experiments were performed with the rate-based network in NEST, each
        network was trained for 2000 epochs of 50ms each. Errors are averaged over 6 independent runs for each
        configuration.}
    \label{fig-dropout}
\end{figure}

To test if the dendritic error model fullfills this requirement, in a first step the self-prediction experiment was
repeated with neuron dropout. To simulate connection probabilities $p_{conn} \in {0.6, 1.0}$, an appropriate number of
synapses was deleted after network setup. To avoid completely separating two neuron populations, this deletion was
performed separately for each of the four synaptic populations.

As expected, removing synapses caused an increase in all four error metrics. Yet even with only 60\% of synaptic
connections present, the network manages to vastly improve from its random initialization. Weight errors are calculated
as mean squared errors over the two matrices, which requires matrices to contain data at every cell. Thus, to compute
these errors, weights of deleted synapses were set to zero in these matrices. This choice was made under the assumption
that a missing connection in an ideal self-predicting network would be matched by a zero-weight - or likewise absent -
synapse. These results indicate that the dendritic error rule is capable of compensating for absent synapses by
correctly identifying and depressing corresponding feedback connections (and vice versa). Through this mechanism, the
network is able to retain its self-predicting properties in spite of physiological constraints.


The extent of this capability was confirmed in SNN, by comparing performance on the Bars dataset of a control network to
a \textit{dropout network}. Both networks were initialized with random synaptic weights, and trained with full
plasticity ($\eta^{ip}_0 = 0.004, \eta^{pi}_0 = 0.01, \eta^{up}_0 = 0.01, \eta^{up}_1 = 0.003$) to best enable the
dropout network to compensate for missing synapses. The dropout network was initialized with $40$ instead of $30$ hidden
layer pyramidal neurons to counteract the deletion of $15\%$ of synapses per synaptic population. Both networks
performed very similarly, with the control network reaching $100\%$ accuracy somewhat faster (Epoch 140 vs. Epoch 200),
while the dropout network exhibited slightly lower test loss at the end of training (results not shown).

These results prove that the dendritic error model is capable of learning in spite of imperfect connectivity, which must
be expected to occur in the cortex. This sets it apart from the previous implementation of a predictive coding network
\citep{Whittington2017}, and further supports its biological plausibility.



\section{Separation of synaptic polarity}


A dogma held in neuroscience for a long time now has been the notion that all neurons are either excitatory or
inhibitory, as dictated by their type (also known as Dale's law \citep{Kandel1968}). Several studies have since shown
that some neurons violate this law through co-transmission or specific release sites for different neurotransmitters
\citep{Svensson2019,Barranca2022}. Despite these findings, pyramidal neurons are still regarded to release exclusively
Glutamate, therefore being strictly excitatory \citep{gerfen2018long,spruston2008pyramidal,Eyal2018}.



A key limitation of the present network model is the requirement that all synapses must be able to assume both positive
and negative polarities. When restricting any synaptic population in the network to just one polarity, the network is
unable to reach the self-predicting state \todo{expand?}. Thus, activity in any neuron must be able to have both
excitatory and inhibitory postsynaptic effects facilitated by appropriate synaptic weights. This requirement is at odds
with biology, which dictates a singular synaptic polarity for all outgoing connections of a neuron, determined by neuron
type and its corresponding neurotransmitter \citeme.


To investigate to what degree the plasticity rule can deal with this constraint, an experiment was conducted: A
population of pyramidal neurons $A$  was connected to another population $C$ with plastic synapses that were constrained
to positive weights. In order to facilitate the required depression, $A$ was also connected to a population of
inhibitory interneurons $B$ through excitatory synapses with random and non-plastic weights. The interneurons in turn
were connected to $C$ through plastic, inhibitory connections. All incoming synapses at $C$ targeted the same dendritic
compartment. When inducing a dendritic error in that compartment, all plastic synapses in the network collaborated in
order to minimize that error. When injecting a positive basal error for example, the inhibitory weights ($C \rightarrow
    B$) decayed, while excitatory synaptic weights ($A \rightarrow B$) increased. Flipping the sign of that error injection
had the opposite effect on weights, and likewise cancelled the artificial error. This shows that a separation of
synaptic polarity does not interfere with the principles of the Urbanczik-Senn plasticity when depression is facilitated
by interneurons.

\begin{figure}[h]
    \centering
    \begin{minipage}{0.2\textwidth}
        \textbf{a)}\par\medskip
        \centering
        \includegraphics[width=0.9\textwidth]{fig_exc_inh_network}
    \end{minipage}\hfill
    \begin{minipage}{0.7\textwidth}
        \textbf{b)}\par\medskip
        \centering
        \includegraphics[width=0.9\textwidth]{fig_exc_inh_split}
    \end{minipage}
    \caption[Error minimization under biological constraints on synaptic polarity and network connectivity]{Error
        minimization under biological constraints on synaptic polarity and network connectivity. \textbf{a)} Network
        architecture. An excitatory population $A$ connects to a dendrite of Neuron $C$ both directly and through
        inhibitory interneuron population $B$. Only synapses $A\rightarrow C$ and $B \rightarrow C$ are plastic through
        dendritic error rules. Populations $A$ and $B$ are fully connected with random weights. \textbf{b)}
        \textit{Left:} All plastic synapses arrive at apical dendrites and evolve according to Equation
        \ref{eq-delta_w_pi}. \textit{Right:} Identical network setup, plasticity for synapses at basal dendrites
        (Equations \ref{eq-delta_w_up}, \ref{eq-delta_w_ip}). \textit{Top:} Dendritic error of a single target neuron.
        Errors of opposite signs are induced at $0$ and $500ms$ (vertical dashed line). \textit{Bottom:} Synaptic
        weights of incoming connections. All initial synaptic weights and input neuron activations were drawn from
        uniform distributions.}
    \label{fig-exc-inh-split}
\end{figure}

Yet, as criticised previously \citep{whittington2019theories}, the one-to-one connections between $A$ and $B$ are
untypical for biological neural networks \citeme. Hence, a second experiment was performed, in which $A$ and $B$ were
fully connected through static synapses with random positive weights. This decrease in specificity of the connections
did not hinder the error-correcting learning, as shown in Fig. \ref{fig-exc-inh-split}.

These results are useful, as they enable a biologically plausible way for excitatory long-range pyramidal projections to
connect to pyramidal neurons in another layer of the network (i.e. in a different part of the cortex). The steps
required to facilitate this type of network are rather simple; A pyramidal neuron projection could enter a distant
cortical area and spread its axonal tree \phrasing within a layer that contains both pyramidal- and inhibitory
interneuron dendrites. If these interneurons themselves connect to the local pyramidal population, Dendritic errors with
arbitrary signs and magnitudes could be minimized.

While error minimization is important, it does not necessarily imply that synaptic credit assignment is successful
aswell. Numerous weight configurations are concievable which could silence dendritic errors, but likely only a small
subset of them is capable of transmitting useful information. To prove that this nonspecific connectivity is compatible
with learning of complex tasks, it was introduced into the dendritic error network. The connection between Interneurons
and Pyramidal neuron apical dendrites was chosen for the first test, as the employed plasticity rule had proven most
resilient to parameter imperfections previously. A network of rate neurons was set up and parametrized as described in
Section \ref{sec-le-tpres} ($t_{pres}= 50ms$). The Weights $w^{pi}$ were redrawn and restricted to postive values, and a
secondary inhibitory interneuron population was created and fully connected to both populations as described in Fig.
\ref{fig-exc-inh-split}. The inhibitory interneuron population was chosen to be 4 times as large as the target pyramidal
population, and $30\%$ of incoming excitatory connections were randomly deleted. The idea behind this was, to seed
interneurons which were to serve as inhibitory counterparts for individual excitatory partners. From this seeding, the
dendritic error rule could then ideally derive useful information about presynaptic activity.

The experiment was successful, as the network was able to learn successfully in competetive time (100\% accuracy after
200 Epochs) albeit to a higher final test loss (results not shown). These results show that the dendritic plasticity
rule is capable of correctly assigning credit to two separate populations under much less sanitized inputs. Further
experiments are required to show \textbf{A:} how large such an inhibitory interneuron population needs to be, and what
role the dropout has to play, \textbf{B:} whether this capability extends to the spiking implementation and \textbf{C:}
if all neuron populations in the network can be connected in this way to separate excitatory and inhibitory pathways.
Such experiments would allow for a closer investigation into how well the dendritic error network corresponds to
cortical connectivity - if at all. Furthermore, the added interneuron populations would themselves have to have some
cortical equivalent which they are to represent.


\subsection{Interneuron nudging}

An easily overlooked connection of this network is the nudging signal from pyramidal neurons to their interneuron
sisters. These were deliberately not included in the previous dropout studies. If any interneuron was to not receive its
nudging signal, its incoming synapses would be unable to adapt their weights. As a result, both interneuron- and
Feedforward weight error would fail to converge, in turn impeding apical error reduction. These one-to-one connections
can  therefore be considered the most important communication channels in the network. If there is no redundancy in the
neurons, the deletion of any of them breaks the network's learning scheme. Sacramento et al. claim that the interneurons
of the network resemble somatostatin-expressing (\textit{SST}) neurons. This is a reasonable assumption, as SST cells
are ubiqutous in the cortex and inhabit the same layers as pyramidal neurons. Furthermore, they share dense and
recurrent synaptic connections to these pyramidal neurons \citep{urban2016somatostatin}. Finally, they have been shown
to receive top-down instructive signals, which have been hypothesized to transmit prediction errors
\citep{Leinweber2017}.

Several experiments similar to those on synaptic polarity were conducted in an attempt to replace these one-to-one
connections with more plausible connectivity schemes. Regrettably, none of them were able to retain the learning
capability of this network. Thus, these connections remain as perhaps the biologically most implausible aspect of the
dendritic error network. Further work is required to investigate if and how this constraint can be relaxed.



\section{Performance of the different implementations}\label{sec-benchmark}

As stated in \citep{Haider2021}, simulating large dendritic error networks with the full leaky dynamics quickly becomes
unfeasable. While the NEST simulator can be regarded as rather fast \citep{albada2018performance}, simulations on it by
design cannot employ batched matrix multiplication, as is typical in machine learning. Thus, by computing neuron updates
individually even in highly structured networks like this one, NEST was expected to perform worse than previous
implementations using PyTorch and dedicated GPUs. Yet not only did the NEST implementations compute rather slowly, the
spiking variant performed worst across the board. To investigate the extent of this, as well as possible causes, several
benchmark experiments were performed. These tests were run on an \textit{AMD Ryzen Threadripper 2990WX} using 8 cores at
up to $3.0GHz$. All reported simulation times $t_{sim}$ are averaged over $5$ independent runs, and only measure the
time taken simulating without considering network initialization. \newline


\begin{figure}[h]
    \centering
    \includegraphics[width=0.85\textwidth]{fig_benchmark_n_hidden}
    \caption[Benchmark of the three implementations under different network sizes]{Benchmark of the three
        implementations under different network sizes. Networks of $[9, n_{hidden}, 3]$ neurons per layer  were
        instantiated with the same synaptic weights and trained for a single epoch of 10 stimulus presentations of
        $50ms$ each. $n_{hidden}=30$ was chosen as a baseline, as it is the default throughout all simulations on the
        Bars dataset.}
    \label{fig-benchmark-n-hidden}
\end{figure}


To compare how network size affects simulation time, all three implementations created for this project were trained on
10 examples of the bars dataset with different numbers of hidden layer pyramidal neurons. The result of this comparison
is shown in Fig. \ref{fig-benchmark-n-hidden}.  The NEST implementation using rate neurons performed best in terms of
speed across the board. This result was slightly surprising, as the demand on the communication interface between
threads is very high, since all neurons transmit an event to each of their postsynaptic targets at every time step.

The NumPy variant is an outlier, and only listed here for completeness. It is the only variant running on a single
thread due to a limitation of NumPy. This could feasibly be improved greatly by using batched matrix multiplications, as
are provided for example by \texttt{PyTorch}. The original implementations do this, but for practical reasons the
Backend was changed here. Notably, this variant exhibits very little slowdown in response to an increase in network
size. It seems, that the vectorization of updates on a single thread scales better with network size than the
event-based communication performed by NEST.

Not only is the spiking variant of this model slower than the rate version, it also scales worse with network size.
Simulation time between $100$ and $250$ hidden layer neurons doubled, compared to an increase of $1.6$ for the rate
network. The Difference between the two was even greater when simulating on an office-grade processor (\textit{Intel
    Core i5-9300H} @ $2.40GHz$, results not shown). Several insights about the comparatively poor performance can be deduced
from a first approximation: The most likely causes for increased compute speed are the communication of events and the
synaptic plasticity rules. Updates to the neuron state are unlikely to be responsible for the worse performance, as both
neuron models are modelled almost identically. These assumptions were tested experimentally.



\begin{figure}[h]
    \centering
    \includegraphics[width=0.7\textwidth]{fig_benchmark_plasticity}
    \caption[Benchmark of both NEST implementations with plastic and non-plastic synapse types]{Benchmark of both NEST
        implementations with plastic and non-plastic synapse types. Deep networks of $300-200-100-10$ pyramidal neurons
        per layer were stimulated with 5 samples of random input $\in\{0,1\}$ for $10ms$ each. synaptic weights were
        initialized between $\{-0.1, 0.1 \}$ to avoid overstimulation of individual neurons. In the plastic paradigm,
        all synapses except for feedback weights $w^{down}$ were plastic with very low learning rates $\eta =
            10^{-10}$.}
    \label{fig-benchmark-plasticity}
\end{figure}


To assess the impact of synaptic updates on computation time, both variants were simulated once with plastic, and once
with static synapses. The simulation environment is set up to model synaptic populations with zero-valued learning rates
as non-plastic synapses (\texttt{static\_synapse} and \texttt{rate\_connection\_delayed} respectively). Thus, by setting
learning rates to zero, it was possible to simulate an entire network without spending any time on synaptic updates.
Results of this experiment are shown in Fig. \ref{fig-benchmark-plasticity}.

As expected, synaptic updates in the spiking network are responsible for a much larger proportion of total simulation
time than in the rate network. A much less anticipated result was that spiking networks are considerably slower even
when plasticity is turned off. This is surprising, as neuron models are almost identical with the exception of some
added complexity in the spike generation process. This added complexity includes drawing from a poisson process, which
might be time-costly depending on the underlying implementation. Another possible reason might be added complexity
associated with SpikeEvents in general, which update some postsynaptic variables not employed for this model. Further
work is required to more rigorously determine the reasons for this poor performance.\newline

\noindent To investigate the degree to which synaptic plasticity and spike transmission in general contribute to
computational cost, two more experiments were conducted. Training durations under different values for the scaling
parameter $\psi$, as well as with different numbers of threads were recorded. Results are shown in Fig.
\ref{fig-benchmark-threads-psi}.

\begin{figure}[h]
    \centering
    \begin{minipage}{0.5\textwidth}
        \textbf{a)}\par\medskip
        \centering
        \includegraphics[width=0.9\textwidth]{fig_benchmark_threads}
    \end{minipage}\hfill
    \begin{minipage}{0.5\textwidth}
        \textbf{b)}\par\medskip
        \centering
        \includegraphics[width=0.9\textwidth]{fig_benchmark_psi}
    \end{minipage}
    \caption[Benchmarks for the spiking implementation]{Benchmarks for the spiking implementation. \textbf{a)}
        Simulation on the MNIST dataset for a network of $784-300-100-10$ neurons and $\psi=250$ on different numbers of
        threads. 10 samples were presented for $50ms$ each, and weights were drawn randomly from a uniform distribution
        $ \{-0.1, 0.1\}$. \textbf{b)} Training of a default network on the Bars dataset using different values of
        the scaling parameter $\psi$. All simulations use 8 threads.}
    \label{fig-benchmark-threads-psi}
\end{figure}

Simulating a large network on an increasing number of threads highlights the diminishing returns gained by spreading out
the simulation of NEST neurons. While initial speedup is high, at some point the benefit of parallelizing neuron updates
is counteracted by the need to communicate more events across threads. It is to be expected that for even higher
parallelization simulation time will begin to increase for this network size.

The second figure shows that reducing $\psi$ much lower will likewise lead to diminishing returns. On the other hand,
increasing it in the hopes of improving learning performance comes at a stark cost to simulation time. This results
should inform future experiments on increasing efficiency through parametrization.

\section{Pre-training}

One of the two major criticisms of the network noted in \citep{whittington2019theories} is the requirement for
pre-training (cf. Supplementary Table \ref{tab-wb-models}). By this, the authors mean the initialization to the self-predicting
state from which most simulations are started. The original paper implicitly considers three different learning
configurations: In the first one, the network starts from the self-predicting state and only feedforward weights are
plastic (cf. Sec. \ref{sec-le-tpres}). In the second one, the network starts from random weights and $Intn \rightarrow
    Pyr$ synapses are plastic, so they can minimize feedback error. The third variant, in which feedback $Pyr \rightarrow
    Pyr$ weights are plastic, is not considered here. An experiment was conducted comparing performance of the first two
variants while training on the Bars dataset. This experiments showed that training was marginally slower, but led to
identical loss (results not shown). While initializing the network to a self-predicting state does give it a slight
'head-start', this is by no means a condition for learning.

Furthermore, training the network towards the self-predicting state does not require any kind of structured input, let
alone targets for activation. The network is driven towards this state purely by noise injection at the input layer. As
background noise is trivial to generate (perhaps unavoidable) for any cortical circuit, the self-predicting state might
be the default rather than the exception. For these reasons, I do not consider pre-training to be an issue that
interferes with the model's biological plausibility.


\section{Behavioral timescale learning}

As a final experiment, the extent to which the network can handle learning on biological timescales was investigated.
One critcism occasionally aimed at Backprop is the requirement for instructive signals to be available immediately
\citep{Bartunov2018}. The assumption is, that an agent in the real world would would select an action, and be informed
about the consequences only after some delay. Learning algorithms should therefore be capable of handling delayed
instructive signals.

Furthermore, all membrane potentials and synaptic weight derivatives of the dendritic error network are reset after each
stimulus presentation. This procedure ensures that residuals from the previous run do not interfere with learning of a
subsequent stimulus. It was confirmed experimentally that networks fail to learn when this reset is not performed after
every training sample (results not shown).

Two additions were made to the model to confirm that it is capable of learning without these constraints. First, the
target activation was delayed to be first injected $5ms$ after the stimulus. This serves to ensure that  that learning
does not rely on simultaneous presentation of stimulus and target. Secondly, instead of manually resetting membrane
potentials, the network was allowed to relax after each training sample. During this relaxation period  (termed
\textit{soft reset}), the network is simulated for $15ms$ without any current injections. A training comparison between
a vanilla network and these two additions is shown in Fig. \ref{fig-idle-time}.


\begin{figure}[h]
    \centering
    \includegraphics[width=0.9\textwidth]{fig_idle_time}
    \caption[Comparison of learning under minimal external control]{Comparison of learning under minimal external
        control. \textbf{Blue:} default parametrization for the Bars dataset. \textbf{Orange:} during the first $5ms$ of
        a training pattern, no target is provided. Afterwards training continues as usual for the remaining $45ms$.
        \textbf{Green:} Additionally, the network is not manually reset after each training sample, but simulated for
        another $15ms$ without stimulation. Increased loss of the delayed target paradigms might be explained by the
        shorter effective training time per stimulus.}
    \label{fig-idle-time}
\end{figure}

All paradigms lead to equally fast learning of the Bars dataset, with delayed target presentation causing a slightly
higher test loss. These results show that constraints like this have only miniscule impact on learning performance of
the dendritic error network. Thus, a cortical network of this kind can be expected to be indifferent to idle time in
which it is only driven by white noise. Likewise, incoming sensory information in the self-predicting state does not
cause plasticity which would drive weights away from what was previously learned. Only when a target is presented to the
output layer will weights adapt. This insight shows that the network requires even less external control, which might be
of use for improving its efficiency \todo{ref outlook}. More importantly, with the need to manually reset membrane
potentials, another biologically implausible mechanism can be omitted from simulations of this network.

It should be noted that this experiment makes the assumption that the brain is either capable of retaining an input
sequence until feedback is available, or otherwise 'replay' the pattern at a later point. While both of these mechanisms
are not completely implausible, they are much less elegant than trace-based solutions for delayed reward signalling
\citep{bellec2020solution}. This constraint should be viewed critically.

\chapter{Discussion}


\section{Contribution}\label{sec-contribution}

In this project, the capabilities and limitations of the dendritic error network and its underlying plasticity rule were
further tested. While sensitive to certain parameter changes, the network was shown to be exceptionally capable of
handling various constraints that affect biological neural networks. Furthermore, the performed experiments can be
interpreted as dispelling some criticisms aimed at the model's biological plausibility. In the following section, 
many of the major questions about the biological plausibility of Backprop from the relevant literature are summarized.


\subsection*{Evaluation of biological plausibility}

The original dendritic error network by design solves several biologically implausible mechanisms of Backprop. It
\textit{\textbf{locally computes prediction errors}} and encodes them within membrane potentials of pyramidal neuron
apical dendrites. Furthermore, it provides two separate solutions to the \textit{\textbf{weight transport problem}}:
First, it is capable of learning through Feedback Alignment, as was done in all present simulations. Secondly,
experiments employing steady-state-approximations have been successful in training feedback weights through variants of
the dendritic plasticity rule. An often overlooked property of biological networks is that feedback signals have an
immediate impact on a neurons output \citep{Larkum2009,Gilbert2013}. This does not occur in classical Backprop, but is
an essential feature of the dendritic error network. Finally, the network relies strictly on \textit{\textbf{Local
plasticity}} \citep{Whittington2017}, and models a (somewhat limited) variability in cell types
\citep{Bartunov2018}.

The present work further improves on the neuron model by showing that spike-based communication does not interfere with
the dendritic plasticity rule, or the intricate balance of excitation and inhibition demanded by the network.
Experiments also showed that the network can be trained with absolutely \textit{\textbf{minimal external control}}
\citep{Whittington2017}. The network requires no external interference such as manual resets or phased plasticity, and
can handle background noise in between sample presentations. Exploratory experiments were conducted in support of the
hypothesis that the plasticity rule is capable of credit assignment when the network conforms strictly to
\textit{\textbf{Dale's law}} \citep{Bartunov2018}. In related experiments, the network proved very capable of
Backprop-like learning when constrained by a more \textit{\textbf{plausible architecture}} \citep{Whittington2017} in
which neuronal populations were connected imperfectly. Finally, the criticism that the network requires pre-training
\citep{whittington2019theories} was found to be largely immaterial. The sum of these observations arguably makes the
dendritic error model one of the most biologically plausible approximations of Backprop yet. 


\section{Limitations}

In spite of these advances, several critical limitations remain. Some general concerns regarding Backprop were
deliberately not addressed in this thesis; It still remains unclear how an agent would come by the labels with which it
might perform this type of Backprop approximation \citep{Bengio2015}. For this reason, some researchers question whether
brains engage in any kind of supervised learning at all \citep{magee2020synaptic}. 


No attempts have yet been made to train the model on anything other than static inputs presented for $10-500ms$.
Therefore, it remains to be seen whether the model is capable of handling the kind of temporal variations or sequences
of patterns which the cortex is required to process \citep{Yamins2016}.

\subsubsection*{Nudging signals}

The requirement for one-to-one connections between pyramidal- and interneurons could not be overcome in this thesis.
Thus, one of the major criticisms from \citep{whittington2019theories} remains unaccounted for. It should also be noted
that these nudging connections transmit somatic activation without any kind of nonlinearity or delay, further making
them biologically questionable. This property is likely the largest limitation of the model, and further work is
required if it is to be addressed. While most general criticisms of Backprop do not apply, the model in its current
state does not plausibly conform to cortical connectivity due to this singular feature.

\subsubsection*{Response to unpredicted stimuli}
One of the predictions about cortical activity made by predictive coding is an increased network activity in response to
sensory input that violates expectations. A diverse set of studies has since reported behavior consistent with this in
various primate cortical neurons (see Table 1 in \citep{bastos2012canonical} for a review). As the dendritic error
network encodes errors in dendritic potentials instead of neuron activations, experiments showed that it does not
exhibit this property in any of its populations. In fact, overall network activity in response to a stimulus seems to
increase after training. In this way, the model conflicts with empirical data on cortical activity.

\subsubsection*{Spike frequencies}

As discussed in Section \ref{sec-c-m-api}, the network in its current state is unable to learn efficiently for low
values of $\psi$. As a result, the implementation demands physiologically impossible spike frequencies from both
pyramidal- and interneurons. While increasing membrane capacitances did relax this constraint somewhat, this change in
turn demands an increase in presentation time per stimulus. Further work is required to determine if the network is
capable of learning when spike frequencies are as low as reported for cortical neurons.

\subsubsection*{Benchmark datasets}

Training on a benchmark dataset would have been very desirable for comparing the spiking implementation to previous
iterations of the dendritic error network. Yet as noted previously, the full network dynamics are prohibitively
expensive for simulations of large networks. Extrapolating the results from Fig. \ref{fig-benchmark-threads-psi} shows
that training on the MNIST dataset is currently unfeasible. A full training with $5,000,000$ sample
presentations (cf. \cite{Haider2021}) would require over one year on 32 threads (excluding testing and validation) with
the current configuration of parameters.

Experiments with subsampled images and smaller network sizes were conducted. Yet, training for these still took on the
order of several days, making the search for suitable parameters very costly. While I am optimistic about the network's
capability in general, no parametrization was found in time under which the network was capable of learning MNIST. 


The challenge of identifying adequate parameters was hindered by training speed in multiple experiments. As the
implementation of the spiking neuron model took much longer than initially anticipated, time was a limiting factor for
all present experiments. Particularly the simulations described in Sections \ref{sec-func-approx}, \ref{sec-c-m-api} and
\ref{sec-dales-law} should be repeated with much more varied sets of parameters. The reported results are expected
improve from this, and insights to be applicable to experiments on more complex learning tasks. Thus, computational
efficiency remains a serious drawback of this model, and has been a major limiting factor for this thesis.


\section{Future directions}

\subsubsection*{Computational efficiency}

The high computational demands of the network were first reported in the original paper \citep{sacramento2018dendritic}. They were largely alleviated
through steady-state approximations and the addition of Latent Equilibrium. The present SNN implementation reintroduces
this issue, and regrettably exacerbates it considerably. Some improvements can be expected by parameter optimizations
such as lowering stimulus presentation time or spike frequencies. Yet to a degree, decreased speed is an inadvertible
price to be paid for a more exact modelling of neuronal processes. Therefore, any attempt at introducing new properties
of biological neurons must be expected to further increase computational demands. Given the (still very high) level of
abstraction of the developed model paired with its poor speed, this perspective is slightly concerning. Hence, the
model requires rigorous optimization.

Some initial directions for this are provided by the benchmarks performed in this study. The spike-based plasticity rule
for example is highly costly. One possible optimization was already provided by \citep{Stapmanns2021}. The authors
discuss an alternative variant for implementing the dendritic plasticity rule. Instead of a strictly event-based or
time-based update rule, a hybrid algorithm called ``Event-based update with compression`` is presented. This variant
tolerates an increased number of synapse updates, but in exchange removes redundant computations. In initial tests, it
proved particularly advantageous for networks in which neurons had a large in-degree. Therefore, this alternative
integration scheme can be expected to perform well for training in the larger networks demanded by more complex
datasets. Regrettably, it was not available in time for this thesis, so potential gains remain speculative.

An alternative improvement to efficiency is approximating the plasticity rule with the instantaneous error at the time
of a spike. This would eliminate the requirement for both frequent updates, and for storing and reading a history of
dendritic error. Thus, a network employing this simplified plasticity rule would be much less computationally costly. As
shown in Fig. \ref{fig-error-comp-le}, error terms in LE networks relax after only a few simulation steps. Thus,
under the condition that only static inputs are considered, this crude approximation is expected to perform fairly well.

The neuron model should likewise be investigated for potential improvements in terms of efficiency. Modeling
interneurons without an apical compartment might yield some improvements (although initial experiments have dampened
expectations for this). It is also possible, that the network does not require integration timesteps as low as $0.1ms$,
which has not been investigated yet.

Finally, further tuning the network's hyperparameters is expected to yield settings that are less computationally
costly. Network relaxation time, stimulus presentation time, spiking frequencies and learning rates form a complex
interplay in this model which has not yet been fully explored. It is therefore to be expected that further optimization
of these might yield networks that can be trained with both lower computational time, and fewer training samples.


\subsubsection*{Neuromorphic hardware}

A different approach which likely would vastly improve simulation speed is a full re-implementation of the model on
neuromorphic hardware. This network fits the self-described niche of such systems almost perfectly; it employs strictly
local plasticity rules, its nodes use leaky membrane dynamics and communicate through binary spikes. By a rough
estimation, even the first generation of Intel's Loihi chips \citep{davies2018loihi} should be capable of simulating
this neuron model. The chip is capable of modelling multiple dendritic trees per neuron, and the learning engine appears
capable of Urbanczik-Senn-like plasticity\footnote{It is possible that the plasticity rule would need to be approximated
somewhat for Loihi 1. The publicly available information about the follow-up chip Loihi 2 \citep{Davies2021} is still
somewhat sparse, but it claims to support a much more diverse set of learning rules.}. Of course, Loihi is only one of
many neuromorphic systems. Another popular system is \textit{BrainScaleS-2}, which appears to be spearheading the field
with regard to simulating segregated dendrites \citep{Kaiser2022}. Regardless of the exact system used, neuromorphic
hardware promises to reduce the high computation time which currently obstructs further research into the model.


\subsubsection*{Neuron model}

Two properties that are part of most spiking neuron models, but have not been investigated here, are membrane reset and
refractory periods. Combined, these modifications would change the spike generation process to that of a stochastic LIF
neuron. Such neuron models have previously performed well for modelling sensory representations in the cortex
\citep{Pillow2008}. Another neuron property considered in that study is \textit{spike-frequency-adaptation}. Neurons
with this mechanism increase their threshold potential in response to previous activity. Such adaptability has been
observed in $\sim 20 \% $ of neurons in the mouse visual cortex \citep{allen2018}, and has been shown to significantly
improve performance in recurrent SNN \citep{bellec2018long,bellec2020solution}. These three changes together would
significantly improve correspondence of the neuron model to physiological data, while potentially also improving their
computational power.

Another point which has not yet been reviewed in terms of biological plausibility is the prospective firing rate in LE
neurons. Haider et al.\ claim that it ``represents a known, though often neglected feature of (single) biological
neurons'' \citep{Haider2021}. They partly base this assessment on the fact that neurons show an increased sensitivity to
coincident spikes through Na channel responses \citep{Platkiewicz2011}. Further work is required to determine whether
prospective activations - particularly as a basis for spike probabilities - appropriately model such processes.  

\subsubsection*{Plasticity rule}

The model of the dendritic trees described here is very rudimentary, which has implications for the plasticity
mechanism. While the Urbanczik-Senn plasticity has been argued to be a type of Hebbian learning
\citep{gerstner2018eligibility,urbanczik2014learning}, it leads to substantially different weight changes. Reviewing the
literature further yielded no arguments that the Urbanczik-Senn plasticity relies on any biologically implausible
mechanisms
\citep{magee2020synaptic,Lillicrap2020,Poirazi2020,sacramento2018dendritic,guerguiev2017towards,Marblestone2016}. Yet
given the extensive amount of data in support of STDP
\citep{magee2020synaptic,gerstner2018eligibility,Bengio2015,Marblestone2016}, the burden of proof is on the dendritic
plasticity rule to override this dogma. This includes finding brain networks which can be modeled by using it, as have
been identified repeatedly for STDP. A fruitful approach might be to investigate STDP in multi-compartment models which
simulate dendritic spikes and plateau potentials in search of similar plasticity dynamics. Exciting advances in this
direction have recently been reported \citep{Bono2017,Schiess2016,magee2020synaptic}, and could potentially be
integrated into the present neuron model.


\subsection*{Cortical circuitry}

The circuitry surrounding the dendritic error network is very simplistic compared to the intricate networks of cortical
microcircuits - not to mention its numerous connections to other parts of the brain. In this regard, the network still
holds much room for improvement. Furthermore, a seminal model for how the cortical microcircuit might be able to perform
predictive coding \citep{bastos2012canonical} appears to conflict with the dendritic error model. While the resulting
network has not yet been computationally modeled, it is backed by a vast amount of empirical data and regarded highly in
the literature \citep{Lillicrap2020,Park2013,whittington2019theories}. One important hypothesis made by it, is that prediction errors are encoded in separate neuronal
populations, rather than dendritic compartments. This claim is shared by other works
\citep{Hertaeg2022,Whittington2017}, and further work is required to find out if the two hypotheses can be reconciled.

A first step towards an integration of the dendritic error model and the cortical circuitry is provided in this thesis.
Present experiments show that the plasticity rule is capable of assigning credit indirectly when dale's law is upheld
via additional interneuron populations. Extending this principle to all sets of synapses in the network would introduce
novel interneuron populations demanding to be identified. The extent to which the resulting network is compatible with
cortical circuitry could prove valuable for further judging the plausibility of the dendritic error model.

\subsubsection*{Deep Feedback Control}

A completely novel solution to the credit assignment problem is provided by \textit{Deep Feedback control}
\citep{Meulemans2021,Meulemans2022}. Instead of approximating Backprop, this algorithm performs Gauss-Newton
optimization, thus employing a previously unexplored approach to training deep neural networks. While originally
described as a purely mathematical model, it might be considered even more biologically plausible than the dendritic
error network. It considers many features of pyramidal neuron dendrites without being held back by any of the common
Backprop criticisms or the constrained connectivity of the present model. The authors also show that it shares a close
mathematical relationship to the dendritic error network, incorporating some interneuron computations into the pyramidal
neurons. Its connectivity however is therefore even further abstracted from the cortical circuitry than the one
discussed here. Regardless of its exact details, this algorithm provides an important insight which was largely
neglected in this thesis (and perhaps the surrounding niche of the neuroscience community) so far: Backprop is not the
only competitive mechanism for assigning synaptic credit in neural networks. Therefore, focussing too narrowly on this
singular solution might prevent us from considering viable alternatives.



\section{Conclusion}

This project further establishes the dendritic error network as one of the most biologically plausible mechanisms for
approximating the Backpropagation of errors algorithm. As hypothesized, the spiking variant of the Urbanczik-Senn
plasticity is capable of performing well in a much more demanding setting than previously shown. The model furthermore
proved to be largely unhindered by the vast majority of biological constraints which were imposed on it. Particularly
constraints on connectivity and synaptic polarity were shown to impede learning to only minor degrees. The model
overcomes all but a few of the general arguments for claiming that the Backpropagation algorithm could not plausibly be
implemented by networks of biological neurons. Only when investigating the correspondence to cortical microcircuits more
closely does the network exhibit serious limitations. 

The predictive coding hypothesis has had a substantial impact on the recent developments in cognitive science. The
dendritic error network is a promising model for explaining how individual computations of this hypothesis could be
distributed across cortical neurons. Finding a biologically plausible mechanism that replaces (or alternatively
explains) the inter-layer nudging signals would arguably elevate it to a prime contender for explaining supervised
learning in the cortex. Nevertheless, such optimism must be paired with restraint given the vast complexity of the
cortex - and associated areas - which a general model would have to encompass. Either way, further research into the
dendritic error network is likely to help us better understand the intriguing capabilities of the human brain.

\chapter{Appendix}



\section{Somato-dendritic coupling}\label{sec-somato-dendr}

\cite{urbanczik2014learning}, discuss a possible extension to their neuron- and plasticity
model, in which the dendro-somatic coupling transmits voltages in both directions. They show
that the plasticity rule requires only minor adaptations for successfull learning under this paradigm. Yet, as described
by passive cable theory, the flow between neuronal compartments is dictated by their respective membrane capacitances.
These are calculated from their membrane areas, which vastly differ in the case of pyramidal neurons. 


15,006
458


will not be considered 
here. The motivation is, that dendritic membrane area is  



\begin{enumerate}
  \item In the torch implementation, there no persistence between timesteps at all. Input is fed into the network and processed feedforward and feedback. Output is read and weights (+biases) are updated. Rinse and repeat.
  \item to what extent should dendritic and somatic compartments decay?
  \item Can (should) we transfer the learned bias from the torch model?
  \item I can "cheat" the apical voltage constraint for self prediction by increasing apical leakage conductance. How does this influence my model?
  \item Is there some analytical approach to identifying why synaptic weights deviate from their intended targets?
  \item I think that lambda needs to be scaled in dependence on $g_{lk}$, such that current inputs, spike inputs and leakage cancel each other out.

  \item How do we deal with population size dependence?

\end{enumerate}

\section{Parameter study}
\begin{itemize}

  \item Transfer function $\phi$
  \item interneuron mixing factor $\lambda$
  \item injected current $I_e$
  \item dendritic leakage conductance $g_{lk,d}$
  \item somatic leakage conductance $g_{lk,s}$
  \item Learning rate $\eta$
  \item synaptic time constants $\tau_{delta}$
  \item noise level $\sigma$
  \item Simulation time $\mathbb{T}$
  \item plasticity onset after the network relaxes
  \item compartment current decay $\tau_{syn}$

\end{itemize}


\subsection*{Observations}

\begin{itemize}
  \item In self-predicting paradigm, Apical errors stay constant, despite interneuron error steadily increasing.
  \item Interneuron error (between neuron pairs) is proportional to absolute somatic voltage in self-predicting paradigm.
  \item abs interneuron voltage is always higher than abs pyramidal voltage. This kind of makes sense, as interneurons receive direct current input proportional to pyramidal voltage in addition to feedforward input. This discrepancy disappears when setting $\lambda$ to 0 as expected.
  \item When plasticity is enabled from a random starting configuration, apical error \textbf{sometimes} converges to better values than can be achieved in both self-predicting paradigms. I believe this to be a huge issue: the self-predicting state does not cause minimal apical voltage, and completely decayed feedback weights are preferable to perfectly counteracting feedback weights.
  \item feedforward weights tend to increase absolutely, i.e. drift towards the closest extreme. \textit{This only happens since I re-implemented the second exponential term in the pyr\_synapse}. Yet they do not simply explode to the nearest extremum, but will traverse a zero weight to reach the maximum with equal sign as the weight they are supposed to match.
  \item feedback weights tend to decay to around zero. Yet they appear to remain close to zero in the direction they are supposed to be.
  \item Idea: I think that the somatic nudging is handled as straight currents being sent to the neuron, instead of the difference between actual and desired somatic voltage.
  \item In the paper and Mathematica code, Feedback learning rate is 2-5 times higher than feedforward lr. In my model, for learning to happen on similar time scales, feedback lr has to be ~100 times lower than feedforward lr. An indicator that my plasticity is messed up.
  \item The simulation is likely producing way too few spikes (5-20 per 1000ms iteration). Could adapting the activation function yield better results?
  \item In the Mathematica solution, leakage conductance is greater than 1! ($\delta U_i = -(g_L + g_D + g_{SI}) U_i + g_D V_{BI} + g_{SI} U_Y$) with $g_L + g_D + g_{SI} = 1.9$
\end{itemize}

\newpage

\chapter{Preliminary structural components}

\section{Synaptic delays}

Where I will inspect the implications of synaptic delays inherent to the NEST simulations on
the model and plasticity rule. In particular, I will look at the biological necessity for this
type of delay and discuss why any model attempting to replicate neuronal processes must be resilient
to these delays.


\section{Literature review - Backpropagation in SNN}

Where I will review other attempts at implementing biologically plausible Backpropagation
alternatives and contrast them to the current model.

\section{NEST Urbanczik-senn implementation}


\section{My neuron model}

\begin{itemize}
  \item Low pass filtering
  \item multi-compartment computation
  \item Imprecision of the ODE
  \item abuse of the somatic conductance
\end{itemize}

\subsection{NEST rate neuron shenanigans}

Given how long I worked on a rate neuron implementation in NEST, some pages should be devoted
to this effort.


\section{My synapse model}

Where I discuss the synapse implementation with regard to multi-compartment neurons,
urbanczik-archiving and in particular the issues with timing that arise from NEST delays.


\section{The relation between the pyramidal microcircuit and actual microcircuits}

Where I can finally use the shit that has been on my whiteboard for half a year...

This will also serve as valuable insight into how plausible this microcircuit actually is,
and might give some insight into possible model extensions.

\subsection{Interneurons and their jobs}

\section{Does it have to be backprop?}

Where I will explain my concerns regarding the usefulness of approximating backpropagation
in light of the substantial one-shot learning capability of the brain and the active inference
model.


\section{Discrepancies between mathematica and NEST model}

1. Weights deviate slightly. This difference can be alleviated by exposing a single stimulus for a longer duration before switching.

\section{Transfer functions}

Where I will discuss the sensitivity of this entire simulation to minor changes in the
parametrization and style of transfer function being used.

\chapter{The weight-leakage tradeoff}

Where I will discuss the issue, that decreasing both synaptic weights and dendritic leakage conductance
lead to more stability in the dendritic voltage, while at the same time requiring longer exposure
per iteration.


\section*{TODOs}

\begin{itemize}
  \item Prove that the network is stable in the self-predicting state and at the end of learning
  \item Show the limits of learning capability (i.e. how big of a network it can match)
  \item Test the network on a real-world dataset (mnist)
  \item prove/find literature on why the poisson process is a rate neuron in the limit.
  \item Does the network still learn when neurons have a refractory period?
  \item Comparison to other spiking backprops
  \item what can we learn from this? does it describe part of the brain
\end{itemize}

\newpage


\chapter{Open Questions}

\begin{itemize}
  \item Any tips for transitioning to large simulation? also regarding the threadripper
  \item Is refractoryness interesting to us or more of a sidenote?
  \item Neuron dropout?
  \item How does one prove that the network is converged and will not diverge again.
  \item randomized/longer synaptic delays?
  \item As a follow up of dropout, maybe even neurogenesis?
  \item Should I look at delaying injection of the target activation?
  \item more ways in which this is biologically implausible?
\end{itemize}

$t_{pres} 10 - 50 \tau$
\bibliography{bib/library.bib}

\end{document}