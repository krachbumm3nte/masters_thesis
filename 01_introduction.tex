
\chapter{Introduction}



\section{Motivation}

\todo{fill with citations?}

The outstanding learning capabilities of the human brain have been found to be elusive
and as of yet impossible to replicate in silicio. While the power and utility of classical
Machine-learning solutions has improved greatly in recent years, these approaches can not serve as 
an adequate model of human cognition. 
The sheer number of neurons and synapses in the brain makes simulations
of an entire brain impossible with current hardware constraints. 
In fact, it has been found to be a substantial challenging to
create artificial neural networks that simulate even parts of human neurophysiology while simultaneously being
able to learn in a goal-oriented way. 



The literature entails numerous approaches to adress this challenges, with varying degrees of success. In this thesis,
I will investigate one such approach, and attempt to modify it in a way that it more closely resembles properties 
exhibited by the human neocortex.

\todo{eigentvalue of the hessian matrix. if they have different signs, something might be off}



\section{The Backpropagation of errors algorithm}

The Backpropagation of errors algorithm (henceforth referred as "Backprop") forms the backbone of modern machine
learning. \citeme It is as of yet unmatched with regard to training in deep neural networks due to its unique capability to
attribute errors in the output of a network to activations of specific neurons within its hidden layers and adapt
incoming weights in order to improve network performance. This property also forms the basis of the algorithm's name;
After an initial forward pass to form a prediction about the nature of a given input, a separate backward pass
propagates the arising error through all layers in reverse order. During this second network traversal, local error
gradients dictate, to what extent a given weight needs to be altered so that the next presentation of the same sample
would elicit a lower error in the output layer.


While Backprop continues to prove exceptionally useful in conventional machine learning systems, attempts use it to
explain the exceptional learning capabilities of the human brain have so far not been successful \phrasing. In fact,
Backprop as a mechanism for synaptic plasticity in the brain is dismissed by many neuroscientists as biologically
implausible. This dismissal is often focussed on three mechanisms that are instrumental for Backprop
\cite{whittington2019theories,Crick1989,Bengio2015}:



\subsection*{Local error representation}

Within conventional artificial neural networks (ANNs), the neurons only serve the purpose of transmitting signals in
a feedforward fashion. The errors on the other hand are computed and propagated by a completely separate algorithm
which can access the entirity of the network state. The algorithm requires the activation of all downstream neurons in
order to compute the weight changes of a given layer. Since plasticity in biological neurons is only dependent on factors
that are local to the synapse, these errors would need to be represented within the neurons of each layer. Several
mechanisms have been proposed to do this in a biologically plausible way, which will be discussed \todo{in a later
  chapter.}


\subsection*{The weight transport problem}

During the weight update stage of Backprop, errors are transmitted between layers with the same weights that are used
in the forward pass. In other words, the magnitude of a neuron-specific error that is propagated through a given connection
should be proportional to its impact on output loss during the forward pass. For this to work, a neuronal network would
require feedback connections that mirror both the network structure and synaptic weights exhibited by the
original network. It was long assumed that the feedback weights are required to be an exact match, but \cite{Liao2016}
showed, that this constraint can be relaxed somewhat to a concordance of weight signs.

Bidirectional connections are common in the cortex, yet it is unclear by which mechanism pairs of
synapses would be able to align. This issue becomes particularly apparent when considering long-range pyramidal
projections, in which feedforward and feedback synapses would be separated by a considerable distance.

Several theories have been proposed as to how biological neural networks could alleviate this issue, which will be 
discussed \todo{in a later section}.

\subsection*{Neuron models}

While the fundamental computational unit of ANNs is called a neuron, it shares little resemblance to biological neurons.
Most notably, these types of artificial neurons transmit a continuous activation. In theory, these activations correspond
to the firing rate of a spiking neuron. Yet particularly with regard to synaptic plasticity, spike based communication
poses a substantial challenge. In the case of backprop, it is even unclear how a derivative of activity can be computed
from a spiketrain.

Furthermore, a given neuron's activation is computed from a simple weighted sum of all inputs. This fails to capture the
complex nonlinearities of synaptic connections that appear to be critical for neuronal computation. Finally, these 
abstract neurons - at least in classical Backprop - have no persistence through time. Thus, their activation is dictated
strictly by the presentation of a single stimulus, in contrast to the leaky membrane dynamics exhibited by biological
neurons.

To differentiate the two, I will be following \cite{Haider2021} in this Thesis; When referring to biologically 
plausible, leaky neurons I will be using the term \textit{"neuronal"}. Respectively, when discussing abstract neurons with
instantaneous response, I will be using \textit{"neural"}.


\todo{discuss supervisor issue?}





\section{Approximating Backprop}


Yet, despite these mechanisms being at odds with biology, when training on real-world data, artificial neural networks
have been shown to learn similar representations as those found in brain areas responsible for comparable tasks
\cite{whittington2019theories,Yamins2016}.

