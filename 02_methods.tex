
\chapter{Methods}

\todo{from \cite{Haider2021}: To differentiate between biologically plausible, leaky neurons and abstract neurons with instantaneous
response, we respectively use the terms “neuronal” and “neural”.}


\section{Neuron and network model}

\begin{figure}
    \centerline{\includegraphics[width={1\linewidth}]{pyramidal.png}}
    \caption{Self-predicting initialization without plasticity}
  \end{figure}
\section{Urbanczik-Senn Plasticity}


The 



One-sided exponential decay kernel

\begin{align}
  \kappa(t) & = H(t)e^{-t/\tau_{\kappa}} \\
  H(t)      & =
  \begin{cases}
    1 & \text{if $t > 0$}    \\
    0 & \text{if $t \leq 0$} \\
  \end{cases}
\end{align}

Antiderivatives:

\begin{align}
  \int_{-\infty}^x H(t)dt = tH(t) = max(0,t)
\end{align}

Convolution:

\begin{align*}
  (f \ast g)(t) & = \int_{- \infty }^{\infty} f(\tau) g(t-\tau) d \tau
\end{align*}
For functions f, g supported on only $[0, \infty]$ (as one-sided decay kernels and spiketrains are), integration limits can be truncated:
\begin{align*}
  (f \ast g)(t) & = \int_{0}^{t} f(\tau) g(t-\tau) d \tau \\
\end{align*}


Plasticity:

\begin{align}
  \frac{dW_{ij}}{dt}(t) & = F(W_{ij}(t), s_i^\ast (t), s_j^\ast (t), V_i^\ast (t)) \\
  F[s_j^\ast, V_i^\ast] & = \eta \kappa \ast (V_i^\ast s_j^\ast)                   \\
  \text{with } V_i^\ast & = (s_i - \phi(V_i )) h(V_i),                             \\
  s_j^\ast              & = \kappa_s \ast s_j.
\end{align}

For an event-based plasticity we need:

\begin{align}
  \Delta W_{ij}(t,T) & = \int_t^T dt' F[s_j^\ast , V_i^\ast ](t')                                                 \\
                     & = \int_t^T dt' \eta \kappa \ast (V_i^\ast s_j^\ast)                                        \\
                     & = \eta \int_t^T dt' \  \int_0^{t'} dt'' \ \kappa(t'-t'') V_i^\ast (t'') s_j^\ast (t'')     \\
                     & = \eta \int_0^t dt' \  \int_{t''}^{t'} dt'' \ \kappa(t'-t'') V_i^\ast (t'') s_j^\ast (t'') \\
\end{align}


Starting with the complete Integral from $t=0$.

\begin{align*}
  \Delta W_{ij}(0,t) & =\eta \int_0^t dt' \  \int_0^{t'} dt'' \ \kappa(t'-t'') V_i^\ast (t'') s_j^\ast (t'')                          \\
                     & = \eta \int_0^t dt'' \  \int_{t''}^{t} dt' \ \kappa(t'-t'') V_i^\ast (t'') s_j^\ast (t'')                      \\
                     & = \eta \int_0^t dt'' \  \left[ \tilde{\kappa}(t-t'') - \tilde{\kappa}(0) \right] V_i^\ast (t'') s_j^\ast (t'') \\
\end{align*}

With $\tilde{\kappa}$ being the antiderivative of $\kappa$:

\begin{align*}
  \kappa(t)         & = \frac{\delta}{\delta t} \tilde{\kappa}(t) \\
  \tilde{\kappa}(t) & = - e^{-\frac{t}{t_{\kappa}}}               \\
\end{align*}

The above can be split up into two separate integrals:
\begin{align*}
  \Delta W_{ij}(0,t) & =\eta \left[ -I_2 (0, t) + I_1(0,t) \right]                                      \\
  I_1(t_1, t_2)      & = - \int_{t_1}^{t_2} dt' \ \tilde{\kappa} (0) V_i^\ast (t') s_j^\ast (t')        \\
  I_2(t_1, t_2)      & = - \int_{t_1}^{t_2} dt' \ \tilde{\kappa} (t_2 - t') V_i^\ast (t') s_j^\ast (t') \\
\end{align*}

Which implies the identities

\begin{align*}
  I_1(t_1, t_2 + \Delta t) & = I_1 (t_1, t_2) + I_1 (t_2, t_2 + \Delta t)                                       \\
  I_2(t_1, t_2 + \Delta t) & = e^{- \frac{t_2 - t_1}{\tau_{\kappa}}} I_2 (t_1, t_2) + I_2 (t_2, t_2 + \Delta t)
\end{align*}


\begin{align}
  I_2 (t_1, t_2 + \Delta t) & = -\int_{t_1}^{t_2 + \Delta t} dt' \ \tilde{\kappa} (t_2 + \Delta t - t') V_i^\ast (t') s_j^\ast (t')                                        \\
                            & = -\int_{t_1}^{t_2} dt' \ \left[ -e^{- \frac{t_2 + \Delta t - t'}{\tau_\kappa}} \right] V_i^\ast (t') s_j^\ast (t')
  -\int_{t_2}^{t_2 + \Delta t} dt' \ \left[ -e^{- \frac{t_2 + \Delta t - t'}{\tau_\kappa}} \right] V_i^\ast (t') s_j^\ast (t')                                             \\
                            & = -e^{- \frac{ \Delta t}{\tau_\kappa}} \int_{t_1}^{t_2} dt' \ \left[ -e^{- \frac{t_2 - t'}{\tau_\kappa}} \right] V_i^\ast (t') s_j^\ast (t')
  -\int_{t_2}^{t_2 + \Delta t} dt' \ \left[ -e^{- \frac{t_2 + \Delta t - t'}{\tau_\kappa}} \right] V_i^\ast (t') s_j^\ast (t')
\end{align}


Using this we can rewrite the weight change from $t$ to $T$ as:


\begin{align*}
  \Delta W_{ij}(t,T) & = \Delta W_{ij}(0,T) - \Delta W_{ij}(0,t)                                               \\
                     & = \eta [-I_2(0,T) + I_1(0,T) + I_2(0,t) - I_1(0,t)]                                     \\
                     & = \eta [I_1(t,T) - I_2(t,T) + I_2(0,t)\left( 1 - e^{- \frac{T-t}{\tau_\kappa}} \right)]
\end{align*}

The simplified \cite{sacramento2018dendritic} case would be:

\begin{align*}
  \frac{dW_{ij}}{dt} & = \eta (\phi(u_i) - \phi(\hat{v_i})) \phi(u_j)                                         \\
  \Delta W_{ij}(t,T) & = \int_t^T dt' \ \eta \  (\phi(u_i^{t'}) - \phi(\widehat{v_i^{t'}})) \  \phi(u_j^{t'}) \\
  \Delta W_{ij}(t,T) & = \eta \int_t^T dt' \  (\phi(u_i^{t'}) - \phi(\widehat{v_i^{t'}})) \ \phi(u_j^{t'})    \\
  V_i^*              & = \phi(u_i^{t'}) - \phi(\widehat{v_i^{t'}})                                            \\
  s_j^*              & = \kappa_s * s_j
\end{align*}


Where $s_i$ is the postsynaptic spiketrain and $V_i^*$ is the error between dendritic prediction and somatic rate and $h( u )$. The additional nonlinearity $h( u ) = \frac{d}{du} ln \  \phi(u)$ is ommited in our model \todo{should it though?}.



\begin{align}
  \tau_l & = \frac{C_m}{g_L} = 10 \\
  \tau_s & = 3
\end{align}

Writing membrane potential to history (happens at every update step of the postsynaptic neuron:

\begin{lstlisting}[language=C++, directivestyle={\color{black}}
                   emph={int,char,double,float,unsigned,exp},
                   emphstyle={\color{blue}}]

UrbanczikArchivingNode< urbanczik_parameters >::write_urbanczik_history(Time t, double V_W, int n_spikes, int comp)
{
	double V_W_star = ( ( E_L * g_L + V_W * g_D ) / ( g_D + g_L ) );
	double dPI = ( n_spikes - phi( V_W_star ) * Time::get_resolution().get_ms() )
      * h( V_W_star );
}\end{lstlisting}

I interpret this as:


\begin{align*}
  \int_{t_{ls}}^T dt' \ V_i^* & = \int_{t_{ls}}^T dt' \  (s_i - \phi(V_i )) h(V_i),               \\
  \int_{t_{ls}}^T dt' \ V_i^* & = \sum_{t=t_{ls}}^T \  (s_i(t) -  \phi(V_i^t ) \Delta t) h(V_i^t) \\
\end{align*}

\begin{lstlisting}[language=C++, directivestyle={\color{black}}
                   emph={int,char,double,float,unsigned,exp},
                   emphstyle={\color{blue}}]
for (t = t_ls; t< T; t = t + delta_t)
{
   	minus_delta_t = t_ls - t;
    minus_t_down = t - T;
    PI = ( kappa_l * exp( minus_delta_t / tau_L ) - kappa_s * exp( minus_delta_t / tau_s ) ) * V_star(t);
    PI_integral_ += PI;
    dPI_exp_integral += exp( minus_t_down / tau_Delta_ ) * PI;
}  
// I_2 (t,T) = I_2(0,t) * exp(-(T-t)/tau) + I_2(t,T)
PI_exp_integral_ = (exp((t_ls-T)/tau_Delta_) * PI_exp_integral_ + dPI_exp_integral);
W_ji = PI_integral_ - PI_exp_integral_;
W_ji = init_weight_ + W_ji * 15.0 * C_m * tau_s * eta_ / ( g_L * ( tau_L - tau_s ) );    
  
kappa_l = kappa_l * exp((t_ls - T)/tau_L) + 1.0;
kappa_s = kappa_s * exp((t_ls - T)/tau_s) + 1.0;
  \end{lstlisting}


\begin{align*}
  \int_{t_{ls}}^T dt' s_j^* & =  \tilde{\kappa_L}(t') * s_j -  \tilde{\kappa_s}(t') * s_j
\end{align*}

$I_1$ in the code is computed as a sum:

\begin{align}
  I_1 (t,T) = \sum_{t'=t}^T \ (s_L^*(t') - s_s^*(t')) * V^*(t')
\end{align}


\section{steady-state potentials in Sacramento (2018)}

\begin{align*}
  u_k^p           & = \frac{g_B}{g_{lk} + g_B + g_A} v^P_{B,k} + \frac{g_A}{g_{lk} + g_B + g_A} v^P_{A,k} \\
  \hat{v}^P_{B,k} & = \frac{g_B}{g_{lk} + g_B + g_A} v^P_{B,k}                                            \\
  \hat{v}^I_{k}   & = \frac{g_B}{g_{lk} + g_B} v^I_{k}                                                    \\
  \lambda         & = \frac{g_{som}}{g_{lk} + g_B + g_{som}}
\end{align*}






\section{Multi-compartment neuron models}



\section{Cortical microcircuits}
