
\chapter{Discussion}


\section{Limitations of the implementation}

Network needs to be reset between stimuli
original does not do that, in NEST it's kindof a big deal.

exposure time and training set still quite large

non-resetting, non-refractory


\section{Whittington and Bogacz criteria}

In which we discuss, to what extent the network conforms to the criteria for biologically plausible learning rules
introduced by \cite{Whittington2017}:
\begin{enumerate}
    \item Local computation. A neuron performs computation only on the basis
          of the inputs it receives from other neurons weighted by the strengths
          of its synaptic connections.
    \item  Local plasticity. The amount of synaptic weight modification is dependent on only the activity of the two
          neurons the synapse connects (and possibly a neuromodulator).
    \item  Minimal external control. The neurons perform the computation autonomously with as little external control
          routing information in different ways at different times as possible.
    \item   Plausible architecture. The connectivity patterns in the model should
          be consistent with basic constraints of connectivity in neocortex.
\end{enumerate}

\section{Should it be considered pre-training?}

\todo{someone said this network needs pre-training and that made me sad :(}


\section{Relation to energy minimization}

\section{Outlook}

This would likely be super efficient on neuromorphics!

I am not going to try plasticity with spike-spike or spike-rate dendritic errors



reward-modulated urbanczik-senn plasticity?

