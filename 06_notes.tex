\chapter{additional notes and questions}

\section{Questions}


\begin{itemize}
    \item how interested are you in the code? reference it, explain it, or shut up?
    \item Code beigefügt als CD? GitHub link? egal?
    \item Citations in my motivation or not?
    \item No defense? is that correct?
    \item Mathematical notation for "variable is changed to" ($W \leftarrow \frac{W}{\psi}$)
    \item Schlechte ergebnisse oder gar keine ergebnisse reporten?
    \item Abgabe postalisch möglich/sinnvoll?
    \item 4 seiten probekapitel ok?
    \item Darf ich auch ein Probekapitel an Johan schicken?
    \item Reconsider title?
    \item Begutachtungsbogen: 2.1 Operationalizations?
\end{itemize}


\section{TODOS}

\begin{itemize}
    \item I can "cheat" the apical voltage constraint for self prediction by increasing apical leakage conductance. How
    does this influence my model?
    \item Does the network still learn when neurons have a refractory period?
    \item talk about feedback plasticity
    \item redo the spike rate experiment, but investigate whether novel stimuli cause bursts.
    \item replace cite with citep
    \item investigate exc-inh split biologically
    \item Urbanczik-senn has little empirical background. shits for the outlook
    \item Spiking network as of yet can only process positive-valued input
    \item test time can be reduced by introducing separate t\_pres
\end{itemize}

squared error / variance of training output = explained variance

\subsection{Interneurons and their jobs}




reciprocal inhibition of SST+ neurons. In agreement,
recent experiments show that feedback input can gate
plasticity of the feedforward synapses through VIP+
neuron mediated disinhibition 176 and that pyrami
dal neurons indeed recruit inhibitory populations to
produce a predictive error177 \citep{Poirazi2020}

\subsection{tpres}

$t_{pres} 10 - 50 \tau$
