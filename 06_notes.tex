\chapter{additional notes and questions}

\section{Questions}


\begin{itemize}
    \item how interested are you in the code? reference it, explain it, or shut up?
    \item Citations in my motivation or not?
\end{itemize}


\section{TODOS}

\begin{itemize}
    \item In the torch implementation, there no persistence between timesteps at all. Input is fed into the network and
    processed feedforward and feedback. Output is read and weights (+biases) are updated. Rinse and repeat.
    \item to what extent should dendritic and somatic compartments decay?
    \item I can "cheat" the apical voltage constraint for self prediction by increasing apical leakage conductance. How
    does this influence my model?
    \item Does the network still learn when neurons have a refractory period?
    \item talk about feedback plasticity
    \item redo the spike rate experiment, but investigate whether novel stimuli cause bursts.
    \item Reconsider title?
    \item replace cite with citep
    \item investigate exc-inh split biologically
    \item Urbanczik-senn has little empirical background. shits for the outlook
\end{itemize}



\subsection{Interneurons and their jobs}




reciprocal inhibition of SST+ neurons. In agreement,
recent experiments show that feedback input can gate
plasticity of the feedforward synapses through VIP+
neuron mediated disinhibition 176 and that pyrami
dal neurons indeed recruit inhibitory populations to
produce a predictive error177 \citep{Poirazi2020}

\subsection{tpres}

$t_{pres} 10 - 50 \tau$
